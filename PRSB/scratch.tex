\begin{table}[!h]
\caption{Data on different aspects of the model for different countries
obtained from reports, peer-reviewed articles and experts' inputs.
\label{tab:countryData}}
\centering
\rowcolors{3}{gray!15}{white} % For this to work, put \PassOptionsToPackage{table}{xcolor} before \documentclass
\begin{tabular}{*{8}{c}}
\toprule
\mrow{Country} & \mrow{Seasons} & \mrow{Production} & \mrow{Consumption} &
\multicolumn{2}{c}{Domestic trade} & \mrow{Processing}& \mrow{International trade} \\
 & & & & Markets & Flows & & \\
\midrule
Bangladesh & \cite{bbs2017} & \cite{bbs2017} & \cite{faostat} & \cite{bbs2017} & -- &
\cite{weinberger2005vegetable,bbs2017} & \cite{EIIndia2015}\\
Cambodia & \cite{sokhen2004,genova2006postharvest,buntong2013} &
\cite{sokhen2004,genova2006postharvest}
&\cite{sokhen2004}&\cite{sokhen2004}&\cite{sokhen2004}&
--&\cite{moustier2007,sokhen2004}\\
Indonesia & \cite{grubben1989,arsanti2015} & \cite{grubben1989,arsanti2015} &
\cite{faostat} & --&-- &-- & \cite{faostat}\\
Laos & \cite{kethonga2004} & \cite{kethonga2004} & \cite{kethonga2004} &
\cite{kethonga2004} & \cite{kethonga2004} & -- &
\cite{moustier2007,kethonga2004} \\
Malaysia &-- & \cite{malaysiaMOA,hengky2016} &\cite{faostat} &-- &-- & -- &
\cite{faostat} \\
Myanmar & -- & \cite{kraas2006megacity,SanSanYi2008} & \cite{Moe2013} &
&\cite{kraas2006megacity} \\
Philippines & \cite{psa2017,batt2011} &  \cite{psa2017} & \cite{concepcion2009} &
\cite{concepcion2011} &  & & {ignored~\cite{concepcion2011}} \\
Singapore & -- & \cite{faostat} & imp.-exp. & city & ignored \\
Thailand & \cite{itharattana1996} & \cite{mict2013,vanitAnunchai2006,itharattana1996}
&\cite{vanitAnunchai2006} &\cite{vanitAnunchai2006,itharattana1996} &
\cite{itharattana1996} & \cite{mict2013}\\
Vietnam & \cite{Vien2003,wijk2007,huong2013} & \cite[Table~16]{wijk2007}
(2003) &
\cite[Table~23]{wijk2007} & \cite{moustier2007,wijk2007,cadilhon2006} & \cite{moustier2007,wijk2007,cadilhon2006} &
\cite{wijk2007,jica2012vietnam} &
\cite{moustier2007,wijk2007} \\
\bottomrule
\end{tabular}
\end{table}

%% ~\cite{pelleg2000x,andrei_novikov_2018_1491324}, an extension of
%% the classical K-means algorithm. It uses Bayesian information criterion to
%% estimate the optimal number of clusters. 
%% 
%% Also, since horticultural products are less
%% durable, owing to lack of good storage and transport
%% infrastructure~\cite{ali2001}, typically, much of the major producing
%% regions are close to cities~\cite{buckmaster2014going}. Only a cell
%% belonging to a locality is affected by the human-mediated pathways. 

%% Since local
%% farm--market dynamics are hard to model~\cite{rebaudo2011}, we used a
%% simple approach.
%% For a cell in a locality (say~$L$), its neighbors are all cells that belong
%% to the locality.  
%% The objective is to cell's state is influenced by the infected cells in
%% its locality through the marketing chain.  In general, it is hard to model
%% the local dynamics as there are several actors in bringing the commodities
%% from farm to market to consumers.  Further, these are country and commodity
%% specific.  See for example Kethonga~et~al.~\cite{kethonga2004} for the
%% typical structure of marketing chains and Rebaudo~et~al.
%% for modeling human interactions in the context of invasive species spread.
%% Here, we use a simple approach. 

%% However, this information was accounted for in experiment
%% design to identify possible starting locations or ``seeds'' for each
%% country.  

%% However,
%% they were accounted for to determine possible routes of entries and in
%% turn, to seed the simulations. However, there is one exception. We included
%% Singapore as a locality of Malaysia while applying the gravity model due to
%% high trade and mobility between the two countries.
%% 
%% case -- Malaysia to
%% Singapore -- by including the latter as a locality of Malaysia and applying
%% the gravity model. While most of the other flows were minor in comparison
%% to domestic flows, major flows that were ignored were from Thailand and
%% Vietnam to Malaysia.
%% Since
%% there is lot of interaction between the two countries, weMalaysia and Singapore
%% was included Malaysian


%% \paragraph{State transitions.} The rules for state transitions are shown in
%% Figure~\ref{fig:SEI}. A susceptible ($S$) cell can be influenced by
%% infectious (state~$I$) cells through the three different pathways. Each
%% such cell infects the susceptible cell with some probability. If \tuta{} is
%% succesfully introduced to a cell, it moves from state~$S$ to~$E$. The
%% exposed state corresponds to the situation where \tuta{} has established,
%% but it is not widespread in the area to influence other cells. It stays in
%% state~$E$ for one time step before transitioning to state~$I$. This is a
%% reasonable assumption considering that the conditions are favorable for the
%% pest to complete a life cycle within one month. 
%%
%% \begin{figure}[ht]
%%     \centering
%%     \includegraphics[height=.16\textwidth]{figs/SEI.pdf}
%%     \caption{Schematic of the SEI process.\label{fig:SEI}}
%% \end{figure}
%%
%% 
%% \paragraph{Susceptibility and infectiousness of a cell.} We have used
%% monthly production to determine the susceptiblity of a cell in state~$S$
%% and infectiousness of a cell in state~$I$. The suitability of a cell~$v$
%% for pest establishment at time~$t$ is denoted by~$\suitable(v,t)$. This
%% is~$1$ if production at~$t$ is non-zero and~$0$ otherwise.  For a cell in
%% state~$I$, the level of infestation in an infected cell~$v$ at time~$t$ is
%% denoted by~$\infest(v,t)$. It is modeled as a linear function of host
%% presence at time~$t$, for which we use the weighted sum of production
%% volume of tomato, eggplant, and potato in that cell
%% at time~$t$. The weights correspond to relative oviposition preference of
%% \tuta{} on the three hosts.
%% 
%%


%% For each region, we obtained the product rate by
%% normalizing quarterly production values with respect to maximum value among
%% these. We used production rate instead of production values since there are
%% several factors that determine a region's production: climate, vegetable
%% preference, demand, etc.  Therefore, it may not be meaningful to compare
%% production across regions.  
%% We conducted a linear regression with the
%% product rate as a dependent variable and precipitation and temperature and
%% elevation as independent variables.  To control elevation, we classified
%% the elevations into two groups, high and low, using $k$-means clustering
%% (SPSS~24.0). Due to the small sample size, we excluded the samples in the
%% high-elevation group and conducted a linear regression analysis for the
%% group of low elevation~($< 235$ masl). 

%% There is very little data
%% available for validation. Most of the data is qualitative, just providing
%% information on growing and harvesting months. The regression function was
%% applied to locations of different countries where this information is
%% available and visually compared.  More details of the methods, country
%% specific data and challenges in this regard are covered in
%% Section~\ref{S:sec:prod}. The infectiousness of cell~$v$ at
%% time~$t$,~$\infest(v,t)$ is the production at~$t$, while its
%% suitability,~$\epsilon(v,t)$ is~$0$ if~$\infest(v,t)=0$ and~$1$ otherwise.
%% on data availability. The first method was used if only seasonal
%% production is available for a region or country. In this case, we first
%% estimated seasonal production in each cell and then disaggregated it
%% to obtain monthly production. Here, due to unavailability of data for tomato and eggplant, we used SPAM's total
%% vegetable production as surrogate. To disaggregate into monthly production,
%% we studied the relationship between production, precipitation, and elevation.
%% For lowland areas, %(elevation less than ???) 
%% production is negatively
%% correlated with precipitation ($r<-0.75$).

%% \paragraph{Network structure.} The resulting network consists of~$8,010$
%% cells and~$109$ localities.  
%%
%%
%% \begin{table}[t]
%%     \caption{Model parameters and variables, their ranges and references.\label{tab:data}}
%%     \centering
%% 	\small
%%     \begin{tabular}{l p{4cm} p{3cm} p{5cm}}
%%     %{cp{.25\textwidth}p{.25\textwidth}lr}
%% 		\hline		
%% 		Parameter & Description & Range/values & Source \\
%% \hline		
%% \hline
%% $\mooreRange$ & Range of Moore neighborhood & 1,2,3 &
%% \cite{guimapi2016modeling}\\
%% $\ell$ & Latency period to transition from $E$ to $I$& 1,2,3 & \\
%% \hline
%% $\beta, \kappa$ & Gravity model parameters & 300--500,2 &
%% \cite{venkatramanan2017towards}\\
%% -- & Locality population threshold and radius & 250,000, 100kms &
%% population of cities~\cite{citypop} and reports\\
%% \hline
%% $\suitable$ & Suitability threshold & 0\\
%% $\infest$ & Infectivity of a cell based on amount of production & & Host
%% production (SPAM), host preference~\cite{sylla2018}, precipitation and
%% elevation \\
%% \hline
%% Scenarios & Seeding simulations: time and cells to infect & 4 cases:
%% Bangladesh (B1 and B2), Malaysia (M1) and Philippines (P1) & \tuta{} incidence
%% reports in Bangladesh, FAOSTAT for international trade and migration
%% reports.\\
%% Start month & & March, April, May & Based on incidence reports\\
%% \hline
%% $\asd$ & Short-distance spread scaling factor & 0--500\\
%% $\afm$ & Local human-mediated dispersal scaling factor & 0--500\\
%% $\ald$ & Long distance spread scaling factor & 0--500 \\
%% \hline
%% \end{tabular}
%% \end{table}
%%

%% In the first part of this section, we present results from the analysis of
%% the spread of \tuta{} in Bangladesh. We demonstrate how accounting for
%% different pathways affects the rate as well as pattern of spread. In the
%% second part, we focus on threat and possible spread in the rest of the
%% study region. The last part focuses on monitoring and control.
%%
%%
%% This section is organized as follows. We first describe the multi-pathway
%% model--our methodological contribution.  This is followed by our results in
%% the context of \tuta{} spread which are summarized as follows: (i)~We
%% identified possible routes by which \tuta{} can invade each country in the
%% region by analysis of current distribution of the pest and its.
%%
%% \paragraph{Assessing the role of each pathway in the spread of \tuta{} in
%% Bangladesh.} 

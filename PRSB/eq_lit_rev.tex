\documentclass[11pt]{article}
%
%%%%%%%%%%%%%%%%%%%%%%%%%%%%%%%%%%%%
%% Common preamble
%%%%%%%%%%%%%%%%%%%%%%%%%%%%%%%%%%%%
% PAGE
\newcommand{\tuta}{\emph{T.~absoluta}}
\newcommand{\infest}{\rho}
\newcommand{\totinf}{\rho_{\mathrm{T}}}
\newcommand{\suitable}{\epsilon}
%% \usepackage{fullpage} % uncomment this when changing to jour/conf style
%% % FONTS
%% \usepackage{lmodern} % enhanced version of computer modern
%% \usepackage[T1]{fontenc} % for hyphenated characters and textsc in section title
%% \usepackage{xr}
%% \externaldocument{supplementary}
\usepackage{microtype} % some compression
\usepackage{times}
\usepackage{setspace} % For double line spacing
\doublespacing
\usepackage[left=1in,top=1in,right=1in,bottom=1in]{geometry}
\usepackage[pagewise]{lineno}
\usepackage{marvosym}
%\linenumbers
% MATH
\usepackage{amssymb}
\usepackage{mathtools} % contains amsmath which comes with align
\usepackage{amsthm} % newtheorem stuff
\usepackage{bm} % for bold math (use $\boldsymbol{}$)
% COLOR
\usepackage[usenames,dvipsnames]{color}
%%
% REFERENCING
% JM: Below is for affiliations
\usepackage{authblk}
% BIBLIOGRAPHY
\usepackage[square,sort&compress,numbers]{natbib} %sorts bibs when they are collectively cited
\usepackage[colorlinks=true,pdfborder={0 0 0},citecolor=RoyalBlue,linkcolor=black,urlcolor=Magenta]{hyperref}
\usepackage{cleveref} %%% IMPORTANT: cleveref comes before \newtheorem commands
   \crefname{figure}{Figure}{Figures}
   \crefname{table}{Table}{Tables}
   \crefname{theorem}{Theorem}{Theorems}
   \crefname{lemma}{Lemma}{Lemmas}
   \crefname{claim}{Claim}{Claims}
   \crefname{section}{Section}{Sections}
   \crefname{observation}{Observation}{Observations}
   \crefname{note}{Note}{Notes}
%%
% TABLES
\usepackage{multirow}
%% \usepackage{ctable} % provides toprule, bottomrule, midrule
\usepackage{array} % new implementation of tabular & array with lots of enhancements
%%
% FIGURES
\usepackage{graphicx}
\graphicspath{./figs}
\usepackage{grffile} % to set right names of files
%%
% CAPTIONS
\usepackage{caption}
\usepackage{subcaption} % supersedes subfigure & subfloat. try using options
%%
% LISTS
\usepackage{enumitem} %\begin{itemize}[leftmargin=*]
%% inline
\newlist{inline}{enumerate*}{1}
\setlist[inline]{before=\unskip{: }, itemjoin={{; }}, itemjoin*={{; and }}, label={(\roman*)}}
% ALGO
\usepackage[ruled,linesnumbered]{algorithm2e}
%% comments & todo
\newcommand{\reportingCells}{\mathcal{C}_R}
\newcommand{\likelihood}{\mathcal{L}}
\newcommand{\aacomment}[1]{({\color{magenta}AA: #1})}
\newcommand{\mmcomment}[1]{({\color{green}MM: #1})}
\newcommand{\tbcomment}[1]{({\color{blue}TB: #1})}
\newcommand{\mrccomment}[1]{({\color{red}MRC: #1})}
\usepackage[colorinlistoftodos]{todonotes}
\newcommand{\TODO}[1]{\todo[inline,color=red!10,size=\small]{#1}}
%% COMMANDS
\newcommand{\comp}[1]{\overline{#1}}  %{\widetilde{#1}}
\DeclareMathOperator{\Var}{Var}
\DeclareMathOperator{\Cov}{Cov}
\DeclareMathOperator{\bigo}{O}
\DeclareMathOperator{\bigom}{\Omega}
\DeclareMathOperator{\davg}{d_{avg}}
\DeclareMathOperator*{\argmax}{arg\,max}
\DeclareMathOperator*{\argmim}{arg\,min}
% Note: \deg is already defined
\newcommand{\expect}{\mathbb{E}}
%
\newcommand{\ceil}[1]{\left\lceil #1 \right\rceil}
\newcommand{\floor}[1]{\left\lfloor #1 \right\rfloor}
%
\newcommand{\reals}{\mathbb{R}}
\newcommand{\field}{\mathbb{F}}
\newcommand{\integers}{\mathbb{Z}}
%
\newtheorem{theorem}{Theorem}[]{\bfseries}{\itshape} 
\newtheorem{lemma}[theorem]{Lemma}{\bfseries}{\itshape}
\newtheorem{claim}[theorem]{Claim}{\bfseries}{\itshape}
\theoremstyle{definition}
\newtheorem{definition}[theorem]{Definition} % {\bfseries}{\itshape}
\newtheorem{observation}[theorem]{Observation} % {\bfseries}
\newtheorem{condition}[theorem]{Condition} % {\bfseries}{\itshape}
\newtheorem{note}[theorem]{Note} % {\bfseries}{\itshape}
% ROMAN NUMERALS
\makeatletter
\newcommand{\rmnum}[1]{\romannumeral #1}
\newcommand{\Rmnum}[1]{\expandafter\@slowromancap\romannumeral #1@}
\makeatother
\begin{document}
\section{Notes}
\begin{itemize}
\item 1. Rapid spread of the invasive yellow‐legged hornet in France: the role of human‐mediated dispersal and the effects of control measures \\

Reaction diffusion model. Nest density $N$
$$\dfrac{\partial N}{\partial t} = D \Big( \dfrac{\partial^2 N}{\partial x^2} + \dfrac{\partial^2 N}{\partial y^2} \Big) + rN \Big( 1 - \dfrac{N}{K} \Big)$$

 Growth rate in year $i$: $r_i$
$$r_i = \ln{\dfrac{n_i}{n_{i-1}}}$$
\item 2. Framework for Modelling Economic Impacts of Invasive Species, Applied to Pine Wood Nematode in Europe\\

Nothing I could find.
\item 3. A Suite of Models to Support the Quantitative Assessment of Spread in Pest Risk Analysis\\

Logistic growth model. Number of invaded cells at time $t$, $n_t$
$$n_t = \dfrac{n_0\text{exp}(rt)}{1+n_0(\text{exp}(rt)-1)/100}$$

\item 4. Human-mediated long-distance jumps of the pine processionary moth in Europe\\

Nothing I could find.
\item 5. Mathematical epidemiology of infectious diseases//

Pg 14. There is an $\ln(s) = r_0...$
Then even better is a probability 
$$F (t) = 1-e^{\int_{t_0}^t \lambda(\tau) d\tau}$$
%6. Equation 11 \\
\item 6. Modeling Infectious Diseases in Humans and Animals\\
Integrating S from -Rt...
\item 7. Effects of Insect-Vector Preference for Healthy or Infected Plants on Pathogen Spread: Insights from a Model\\
$1-(1-f)^{p_i}$
\item 8. SPREAD OF AN INVASIVE PATHOGEN OVER A VARIABLE LANDSCAPE: A NONNATIVE ROOT ROT ON PORT ORFORD CEDAR\\
Probability that it escapes infection. $H(t,x) = h_0(t)\text{exp}(B'x)$
\item 9. Stochastic spread of Wolbachia\\
Eq 1.2, similar I think
\item 10. Modeling malware spreading dynamics\\
Some stuff that looks similar
\item 11. SAS Survival Analysis Techniques for Medical Research\\
Gives survival probability as similar.
\item 14. Mathematical models of infectious disease transmission\\
Gives something for final proportion which are infected as a similar form
\item 15. A THOUSAND AND ONE EPIDEMIC MODELS\\
Gives waiting times as $e^{-\varepsilon t}$, still being exposed after $t$ time steps. $\varepsilon$ has to do with transfer rate. Says 18 showed it was exponential.
\end{itemize}

\section{Summary}
Our function structure is $p=\alpha \epsilon (1-\text{exp}(\text{infectivity}))$. This is an exponential probability distribution. This form of the infection probability function has been seen before in epidemiological models \cite{5,11,14,15}, both continuous and discrete time models. In 5, we see an SI model being investigated mathematically. They consider a continuous time model, and define the probability that an individual will not be infected as:
$$F (t) = e^{\int_{t_0}^t \lambda(\tau) d\tau}\;,$$
where $\lambda$ is the "force of infection". This directly correlates to our probability of leaving the susceptible state. \par

In 11, we see more complexity, but we also see some derivation of this function form from the "proportional hazards model". They similarly consider a continuous time model and an exponential probability function of remaining susceptible. \par 

In 15, we see a discrete time version of this function form, with the probability of still being exposed after $t$ time steps as:
$P(t) = e^{-\epsilon t}$, where $\epsilon$ relates to the "transfer rate". This was shown to be the case when considering traditional SIR models mathematically in 18.

\par 



\end{document}

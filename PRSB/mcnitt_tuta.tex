\PassOptionsToPackage{table}{xcolor}
\documentclass[11pt]{article}
%
%%%%%%%%%%%%%%%%%%%%%%%%%%%%%%%%%%%%
%% Common preamble
%%%%%%%%%%%%%%%%%%%%%%%%%%%%%%%%%%%%
% PAGE
\newcommand{\tuta}{\emph{T.~absoluta}}
\newcommand{\infest}{\rho}
\newcommand{\totinf}{\rho_{\mathrm{T}}}
\newcommand{\suitable}{\epsilon}
%% \usepackage{fullpage} % uncomment this when changing to jour/conf style
% FONTS
\usepackage{textcomp}
\usepackage{lmodern} % enhanced version of computer modern
\usepackage[T1]{fontenc} % for hyphenated characters and textsc in section title
\usepackage{xr}
\externaldocument[S:]{supplementary}
\usepackage{microtype} % some compression
\usepackage{times}
\usepackage{setspace} % For double line spacing
\doublespacing
\usepackage[left=1in,top=1in,right=1in,bottom=1in]{geometry}
\usepackage[pagewise]{lineno}
\linenumbers
\usepackage{marvosym}
%% \newcommand\wordcount{\immediate\write18{texcount -sub=section \jobname.tex  | grep ``Section'' | sed -e 's/+.*//' | sed -n \thesection p > 'count.txt'} (\input{count.txt}words)}
% MATH
\usepackage{amssymb}
\usepackage{mathtools} % contains amsmath which comes with align
\usepackage{amsthm} % newtheorem stuff
\usepackage{bm} % for bold math (use $\boldsymbol{}$)
% COLOR
\usepackage[usenames,dvipsnames]{color}
%%
% REFERENCING
% JM: Below is for affiliations
\usepackage{authblk}
% BIBLIOGRAPHY
\usepackage[square,sort&compress,numbers]{natbib} %sorts bibs when they are collectively cited
\usepackage[colorlinks=true,pdfborder={0 0 0},citecolor=RoyalBlue,linkcolor=black,urlcolor=Magenta]{hyperref}
\usepackage{cleveref} %%% IMPORTANT: cleveref comes before \newtheorem commands
   \crefname{figure}{Figure}{Figures}
   \crefname{table}{Table}{Tables}
   \crefname{theorem}{Theorem}{Theorems}
   \crefname{lemma}{Lemma}{Lemmas}
   \crefname{claim}{Claim}{Claims}
   \crefname{section}{Section}{Sections}
   \crefname{observation}{Observation}{Observations}
   \crefname{note}{Note}{Notes}
%%
% TABLES
\usepackage{multirow}
%% \usepackage{ctable} % provides toprule, bottomrule, midrule
\usepackage{array} % new implementation of tabular & array with lots of enhancements
%%
% FIGURES
\usepackage{graphicx}
\graphicspath{./figs}
\usepackage{grffile} % to set right names of files
%%
% CAPTIONS
\usepackage{caption}
\usepackage{subcaption} % supersedes subfigure & subfloat. try using options
%%
% LISTS
\usepackage{enumitem} %\begin{itemize}[leftmargin=*]
%% inline
\newlist{inline}{enumerate*}{1}
\setlist[inline]{before=\unskip{: }, itemjoin={{; }}, itemjoin*={{; and }}, label={(\roman*)}}
\usepackage{silence}
\WarningFilter{ctable}{Transparency disabled:}
\WarningFilter{xcolor}{Incompatible color}
% ALGO
\usepackage[ruled,linesnumbered]{algorithm2e}
%% comments & todo
\newcommand{\reportingCells}{\mathcal{C}_R}
\newcommand{\similarity}{\mathcal{S}}
\newcommand{\aacomment}[1]{({\color{magenta}AA: #1})}
\newcommand{\mmcomment}[1]{({\color{green}MM: #1})}
\newcommand{\tbcomment}[1]{({\color{blue}TB: #1})}
\newcommand{\mrccomment}[1]{({\color{red}MRC: #1})}
\newcommand{\jmcomment}[1]{({\color{cyan}JM: #1})}
\usepackage[colorinlistoftodos]{todonotes}
\newcommand{\TODO}[1]{\todo[inline,color=red!10,size=\small]{#1}}
%% COMMANDS
\newcommand{\comp}[1]{\overline{#1}}  %{\widetilde{#1}}
\DeclareMathOperator{\Var}{Var}
\DeclareMathOperator{\Cov}{Cov}
\DeclareMathOperator{\bigo}{O}
\DeclareMathOperator{\bigom}{\Omega}
\DeclareMathOperator{\davg}{d_{avg}}
\DeclareMathOperator*{\argmax}{arg\,max}
\DeclareMathOperator*{\argmim}{arg\,min}
% Note: \deg is already defined
\newcommand{\expect}{\mathbb{E}}
%
\newcommand{\ceil}[1]{\left\lceil #1 \right\rceil}
\newcommand{\floor}[1]{\left\lfloor #1 \right\rfloor}
%
\newcommand{\reals}{\mathbb{R}}
\newcommand{\field}{\mathbb{F}}
\newcommand{\integers}{\mathbb{Z}}
\newcommand{\pshort}{p_s}
\newcommand{\plocal}{p_{\ell}}
\newcommand{\pld}{p_{\ell d}}
\newcommand{\asd}{\alpha_s}
\newcommand{\afm}{\alpha_{\ell}}
\newcommand{\ald}{\alpha_{\ell d}}
\newcommand{\produce}{\mathrm{Prod}}
\newcommand{\veg}{\mathrm{V}}
\newcommand{\temp}{\mathrm{T}}
\newcommand{\consume}{\mathrm{Pop}}
\newcommand{\locality}{\mathrm{L}}
\newcommand{\export}{\mathrm{Export}}
\newcommand{\import}{\mathrm{Import}}
\newcommand{\process}{\mathrm{Proc}}
\newcommand{\moore}{\mathrm{M}}
\newcommand{\mooreRange}{r_\mathrm{M}}
%
\newtheorem{theorem}{Theorem}[]{\bfseries}{\itshape} 
\newtheorem{lemma}[theorem]{Lemma}{\bfseries}{\itshape}
\newtheorem{claim}[theorem]{Claim}{\bfseries}{\itshape}
\theoremstyle{definition}
\newtheorem{definition}[theorem]{Definition} % {\bfseries}{\itshape}
\newtheorem{observation}[theorem]{Observation} % {\bfseries}
\newtheorem{condition}[theorem]{Condition} % {\bfseries}{\itshape}
\newtheorem{note}[theorem]{Note} % {\bfseries}{\itshape}
% ROMAN NUMERALS
\makeatletter
\newcommand{\rmnum}[1]{\romannumeral #1}
\newcommand{\Rmnum}[1]{\expandafter\@slowromancap\romannumeral #1@}
\makeatother
%%
% TWO VERSIONS
%% \usepackage{etoolbox}
%% \newtoggle{withappendix}
%% \toggletrue{withappendix} % comment this if you want journal version
%% Usage: iftoggle{withappendix}{}{}
%%
%% math operator
%% \DeclareMathOperator{\sgn}{sgn}
% CODEBOX
%% \usepackage[framemethod=tikz]{mdframed}
%% \newmdenv[linecolor=black!10,innerlinewidth=0pt, roundcorner=4pt,innerleftmargin=6pt,
%% font=\ttfamily,innerrightmargin=6pt,innertopmargin=6pt,
%% innerbottommargin=6pt,backgroundcolor=black!10]{codeblock}
%%%%%%%%%%%%%%%%%%%%%%%%%%%%%%%%%%%%
%% preamble ends
%% from now on, draft specific
%%%%%%%%%%%%%%%%%%%%%%%%%%%%%%%%%%%%
%% \RequirePackage[l2tabu, orthodox]{nag}
\makeatletter
\renewcommand\AB@affilsepx{, \protect\Affilfont}
\makeatother
% Title on edge of suggested length
\title{Assessing the Multi-pathway Threat from an Invasive Agricultural
Pest: \emph{Tuta~absoluta} in Asia}
%% \title{A Multi-pathway Model to Assess the Threat of Invasive Agricultural
%% Pests: Case Study of \emph{Tuta~absoluta} in Asia}
\author[1]{Joseph~McNitt}
\author[2]{Young~Yun~Chungbaek}
\author[2]{Henning~Mortveit}
\author[2]{Madhav~Marathe}
\author[3]{Mateus~Ribeiro~de~Campos}
\author[3]{Nicolas~Desneux}
\author[4,5,6]{Thierry~Br\'{e}vault}
\author[7]{Rangaswamy Muniappan}
\author[2]{Abhijin~Adiga}
\affil[1]{Department of Mathematics, Virginia Tech}
\affil[2]{Biocomplexity Institute \& Initiative, University of Virginia}
\affil[3]{French National Institute for Agricultural Research}
\affil[4]{BIOPASS, CIRAD-IRD-ISRA-UCAD, Dakar, Senegal}
\affil[5]{CIRAD, UPR AIDA, F-34398 Montpellier, France}
\affil[6]{Universit\'{e} de Montpellier, CIRAD, Montpellier, France}
\affil[7]{Feed the Future Integrated Pest Management Innovation Lab}
\date{}
\setcounter{Maxaffil}{0}
\renewcommand\Affilfont{\itshape\small}

\begin{document}
\maketitle

\begin{abstract}
Modern food systems facilitate rapid dispersal of pests and pathogens
through multiple pathways. Complexity of the spread dynamics and data
inadequacy make it challenging to model the phenomenon and prepare for
emerging invasions. We present a generic framework to study the
spatio-temporal spread of invasive species as a multi-scale propagation
process over a time-varying network accounting for climate, biology, seasonal
production, trade and demographic information. Machine learning techniques
are used in a novel manner to capture model variability and analyse
parameter sensitivity. We applied the framework to study the spread of a
devastating pest of tomato, \emph{Tuta absoluta}, in South and Southeast
Asia -- a region at the frontier of its current range. Analysis with
respect to historical invasion records suggests that even with modest
self-mediated spread capabilities, the pest can quickly expand its range
through domestic city-to-city vegetable trade. There is a strong chance
that within five to seven years \tuta{} will invade all major vegetable
growing areas of Mainland Southeast Asia if no steps are taken to mitigate
the spread.  Further, we showed that monitoring and effective interventions
in major production areas could greatly reduce the speed of the spread.
\end{abstract}
%%
\section{Introduction}
As the intensity of trade and human mobility increase, so does the rate of
exotic species invasions~\cite{hulme2009trade}. Climate change and
detrimental impact of intensive agriculture on natural resources further
aggravate this problem~\cite{early2016global}.
Understanding the dynamics of invasive species spread is imperative to the
achievement of zero hunger, no poverty, and good health and well being, which
are among the sustainable development goals drafted by the United
Nations. Models play an important role in predicting the
spatio-temporal spread, identifying roles of different pathways, assessing
efficacy of control strategies and exposing gaps in the understanding of
the phenomenon~\cite{cunniffe2015thirteen,epstein2008model}. However,
impending invasions of agricultural pests present difficult challenges.
Accounting for multiple drivers of dispersal
invariably makes the model complex.  At the same time, data inadequacy
makes it nearly impossible to calibrate and validate these models. Despite
these limitations, a natural goal for a modeller is to provide useful
insights into the mechanism of spread, and thus help design effective
policies for its prevention and mitigation.

Network propagation models have been widely used to study large interacting
biological, social and technical systems. Some examples include infectious
disease spread, online social networks, cascading failures in
infrastructure networks~\cite{barrat2008dynamical}.  Increasingly,
they are being applied to study invasive species
dynamics~\cite{douma2016pathway,carrasco2010unveiling,nopsa2015ecological}.
Unlike pest risk maps generated by species distribution
models~\cite{pearson2007species}, the resulting dynamics of such a
validated model yields a causal description of the underlying complex
system. Here, we present a multi-pathway propagation model to study the
spread invasive agricultural pests. The model accounts for both
self-mediated and human-mediated spread and effectively encapsulate spatial
heterogeneity, temporal variations and multi-scale nature of the
propagation mechanisms.

We applied this framework to study the spread of the South American Tomato
leafminer or \emph{Tuta absoluta}, a pest of the tomato crop and
representative of recent biological invasions that have significantly
perturbed global food production.  Indigenous to South America, \tuta{} was
accidentally introduced to Spain in~2006, and since then it has rapidly
spread throughout Europe, Africa, Western and Central Asia, the Indian
subcontinent, and parts of Central
America~\cite{desneux2010biological,biondi2017}. With tomato being a
commercially important crop, this invasion has had significant global
impact~\cite{campos2017western}. It is well accepted that trade
played a critical role in \tuta{}'s rapid spread. On multiple occasions it
has been discovered in packaging stations and its spread pattern is
correlated with prime trade routes~\cite{karadjova2013tuta}. Our study
region is South and Southeast Asia-- a region at the frontier of its
current range -- comprising of 10 countries: members of the Association of
Southeast Asian Nations (ASEAN) and Bangladesh.  In the recent years, there
has been a thrust to improve vegetable production in all the countries of
this region. With the pest having already spread to major
tomato producing areas in Bangladesh, there is a
high chance that it will be introduced to the remaining countries in the
near future. Such invasions can have devastating effect on the economy and
livelihood of farmers.  Moreover, invasion in Mainland Southeast Asia in
particular is a serious threat to China, the largest producer of tomato,
and Australasian neighbours. To our knowledge, this is the first study that
explicitly considers multiple pathways of introduction and spread of
\tuta{}.  Earlier modelling efforts have only accounted for ecological
aspects and self-mediated
spread~\cite{desneux2010biological,tonnang2015identification,guimapi2016modeling}.

To develop this model, we identified, analysed, and fused disparate
datasets corresponding to climate (temperature, precipitation, elevation),
biology (host preference, suitability, population growth), production,
trade (domestic trade, imports, exports, vegetable processing, human
population). Integrating information from
research articles, annual reports and models, the spatio-temporal
distribution of host crops was estimated. Trade flows were estimated using
a gravity model approach. Since \tuta{} was recently introduced into this
region, incidence reports are few in number and possibly inaccurate.
Machine learning techniques such as clustering and decision-tree algorithms
were used to capture model variability and study parameter sensitivity.

\paragraph{Summary of results.} 
Our analysis with respect to historical invasion records indicate two
possible explanations for the spread of \tuta{} in Bangladesh, one with
trade as the dominant pathway and the other without. The former explanation
suggests that even with slow self-mediated dispersal, the pest could have
rapidly expanded its range aided by domestic tomato trade. Under the
assumption that trade is an important pathway, country specific analysis
shows that once introduced to a major production area, the pest will spread
all over the country within two to three years. Since highly populated
urban centres attract trade inflows, high production regions close to such
areas are particularly vulnerable to attacks. We also showed that
quarantining few key production areas can contain the spread.

%%
\section{Methods}
%%
\paragraph{Data.} The global datasets used in the model and for analysis are
described in Table~\ref{S:tab:data} of the supplement. Country specific data on
seasonal production, consumption, processing and trade was
obtained from websites of agriculture ministries, research articles and technical
reports (Table~\ref{S:tab:countryData} in the supplement). Almost
all the datasets used are openly available. Some information was
provided by local contacts in Bangladesh (for e.g., \tuta{} incidence
reports in Table~\ref{S:tab:bgdData}), Vietnam and Cambodia.
%%
\paragraph{\tuta{} biology.}
The tomato leafminer exhibits a short life cycle of about 24--38 days
(temperature at $25\pm3^\circ$C), from egg to adult, as it is a
multivoltine species with overlapping generations in the
field. It causes serious damage to
numerous solanaceae crops such as eggplants, potatoes, and especially
tomato crops~\cite{sylla2018}. It penetrates into tomato leaves, stems, or
fruits, wherein it feeds and develops by creating conspicuous mines as well
as galleries. Considering the warm weather throughout the year,
particularly in the dry season, the study region presents ideal conditions
for rapid development and spread of \tuta{}. Pest risk
analysis~\cite{tonnang2015identification} shows that the Ecoclimatic Index
for this region is above~$50$ (highly suitable). Spatial distribution
assessment survey of \tuta{} eggs has shown its high dispersive capacity in
tomato producing areas~\cite{martins2018assessing}. The dispersion in a
tomato cultivation starts mainly at the periphery and
the pest is able to migrate between tomato farms to generate egg aggregation at the
crop edges. The pest spread behaviour among seasonal crop resources
is often non-random and directional~\cite{martins2018assessing}.
Sylla~et~al.~\cite{sylla2018} analysed host preference of~\tuta{} in France
and Senegal. While the highest preference is for tomato, it can survive
well on eggplant and potato, which happen to be major vegetable crops in
the study region. However, since \tuta{} primarily attacks leaves of
eggplant and potato, the chance of the pest spreading through trade of
these crops seems to be low.
%%~\cite{guedes2012tomato}
%%
\paragraph{Multi-pathway spread model.} We developed a stochastic
multi-scale propagation model to simulate the multi-pathway spread
of~\tuta{}. Key concepts are illustrated in Figure~\ref{fig:concept}. The
study region is divided into cells by overlaying a grid
(0.25\textdegree~$\times$~0.25\textdegree). Each cell is in one of the
three states: susceptible ($S$) denoting pest free state, exposed ($E$)
denoting that the pest has been introduced but the population has not yet
built up to influence other cells, and infectious ($I$) denoting that the
pest has established and the cell can influence its neighbours. The cell
states are updated in discrete time steps, where each step~$t$ corresponds
to a month.  The probability that a cell~$v$ transitions from state~$S$
to~$E$ is determined by (i)~suitability of the cell for \tuta{} to
establish at that time step~$\suitable(v,t)$ and (ii)~influence of
``neighbouring'' cells in state~$I$ depending on the pathway. An exposed
cell transitions to state~$I$ after a latency period of~$\ell$ time steps.
This is the time required for the population to build up to infect other
cells.  Once the pest has established in a cell, the cell remains infected
forever, a fair assumption considering that, historically, eradication of
\tuta{} has not been successful\footnote{The only exception is United
Kingdom where the pest was detected early and
eradicated.}.  The
infectiousness of a cell~$\infest(v,t)$ is modelled as a linear function of
host presence at time~$t$, for which we use the weighted sum of production
volume of tomato, eggplant, and potato in that cell at time~$t$. The
weights correspond to relative carrying capacity of \tuta{} on the three
hosts~\cite{sylla2018}.
%%~(\url{https://gd.eppo.int/reporting/article-340})
%%
\begin{figure}[t]
\centering
\begin{subfigure}[b]{.4\textwidth}
    \includegraphics[width=1.1\textwidth]{figs/model_schematic.pdf}
\caption{\label{fig:concept}}
\end{subfigure}
%%
\begin{subfigure}[b]{.56\textwidth}
    \includegraphics[width=1.05\textwidth]{figs/pipeline.pdf}
\caption{\label{fig:pipeline}}
\end{subfigure}
\caption{\textbf{The multi-pathway model.} (a)~The network structure,
pathways and dynamics are captured in the illustration.
%% , which is
%% approximately~$27.8\mathrm{km}\times\mathrm{27.8km}$ at the equator. 
%% These
%% dimensions are comparable to that used in the cellular automata model of
%% Guimapi~et.~al.~\cite{guimapi2016modeling} ($25\mathrm{km}\times25\mathrm{km}$). Details of locality construction are in Section~\ref{S:sec:locality}. 
%% Long distance human-mediated dispersal is modeled
%%las the spread between localities through. 
Also shown are the states and factors
that influence state transitions: infectiousness of a neighbour, suitability
of the cell for pest establishment, pathway parameters and latency period.
\label{fig:modelConcept}
(b)~\textbf{Pipeline.} The process of constructing the spatio-temporal
network of cells is outlined. Key modules are highlighted along with 
input data.}
\end{figure}
%%

There are three pathways by which a cell can become infected:
short-distance dispersal, local human-mediated dispersal and long-distance
dispersal. Short-distance dispersal captures the spread through natural
means; from an infested cell to cells in its Moore neighbourhood of
range~$\mooreRange$.  The probability that a susceptible cell gets exposed
(E) at time step~$t$ through short-distance spread is as follows:
%%
\begin{linenomath}
\begin{align}\label{eqn:pshort}
    \pshort(v,t)=\suitable(v,t)\bigg(1-
    \exp\Big(-\asd\sum_{v'\in\moore_v(\mooreRange)}\infest(v',t)\Big)\bigg),
\end{align}
\end{linenomath}
%%
The probability depends on the suitability of the cell~$\suitable(v,t)$,
infestation level of each neighbouring cell in the Moore neighbourhood with
range~$\mooreRange$,~$\infest(v',t)$ and the scaling factor,~$\asd$, which
is the transmission rate for this pathway. The function form is explained
in Section~\ref{S:trans}.

For human-assisted spread we identified large urban areas in the region
which we refer to as {\it localities} (Figure~\ref{fig:modelConcept}) and
considered interactions within and between localities. These areas attract
vegetable flows due to high consumption or production and
house the necessary infrastructure: wholesale markets, traders and
distributors.  Each \emph{locality} consists of all grid cells which are
within a certain distance (determined by \emph{locality radius}) from its
corresponding centre. Local human-mediated dispersal is modelled as the
spread between cells belonging to a locality.  Every cell~$v$ is influenced
by cells in its locality~$\locality$ based on their infectiousness.  The
expression is similar to that in~\eqref{eqn:pshort}, but with cells in the
locality instead of the Moore neighbourhood.
%%
\begin{linenomath}
\begin{align}\label{eqn:plocal}
    \plocal(v,t)=\suitable(v,t)\bigg(1-
    \exp\Big(-\afm\sum_{v'\in\locality}\infest(v',t)\Big)\bigg),
\end{align}
\end{linenomath}
%%
where~$\afm$ is the scaling factor. The details of
locality construction are provided in Section~\ref{S:sec:locality} of SI.

Long-distance human-mediated dispersal corresponds to spread through trade
between localities. For this purpose, we considered only tomato trade as
there is not much evidence of \tuta{} spreading through trade of other
hosts. We modelled domestic trade using a gravity model approach accounting
for tomato production, processing, imports and exports in each locality and
the travel time between localities. 
The probability of spread is directly proportional to the trade
flow~$F_{ij}$ from one locality ($i$) to another ($j$).
Suppose cell~$v$ belongs to locality~$i$. Then, the probability of cell~$v$
transitioning from~$S$ to~$E$ due to long-distance dispersal is given by:
%%
\begin{linenomath}
\begin{align}\label{eqn:pld}
    \pld(v,t)=\suitable(v,t)\bigg(1-
    \exp\Big(-\ald\sum_{j\ne i}\sum_{v'\in\locality(j)}F_{ji}\infest(v',t)\Big)\bigg),
\end{align}
\end{linenomath}
%%
where~$\ald$ is the scaling factor.
The model parameters and their values are summarised in
Table~\ref{tab:param}. 
%%
\begin{table}[t]
\caption{Model parameters and their values.\label{tab:param}}
    \centering
	\small
\rowcolors{2}{white}{gray!15} % For this to work, put \PassOptionsToPackage{table}{xcolor} before \documentclass
    \begin{tabular}{p{.07\textwidth}p{.39\textwidth}p{.47\textwidth}}
    %{cp{.25\textwidth}p{.25\textwidth}lr}
		\hline		
		Parameter & Description & Value/range \\
\hline		
\hline
$\mooreRange$ & Range of Moore neighbourhood & $\{1,2,3\}$ corresponding to
spread per month of 
approximately~$25$km,~$50$km and~$75$km
respectively~\cite{guimapi2016modeling,martins2018assessing}. \\
%% $\suitable$ & Suitability threshold \\
$\ell$ & Latency period to transition from $E$ to $I$ & $\{1,2,3\}$ months
based on time for the pest to complete life cycle (\tuta{} biology in
Methods). \\
season & Disaggregation of annual production to monthly values
& \emph{Uniform} throughout the year or \emph{seasonal} based on regression
analysis (Methods). \\
$\beta$ & Gravity model distance function exponent & $\{0,1,2\}$ \\
$\kappa$ & Gravity model distance function cut-off & Between $4$ to $16$ hours
of travel time. \\
seed & Location and time of initial infestation & Scenarios based on
countries (Table~\ref{S:tab:seeds})\\
locality radius & Determines cells assigned to a locality & 100km \\\hline
$t_s$ & Time step of initial infestation & $\{3,4,5\}$ corresponding
to March, April and May respectively based on first report in
Bangladesh~\cite{hossain2016first}. \\
%% $\infest$ & Infectivity of a cell based on amount of production & \\
$\asd$ & Short-distance spread scaling factor & In the interval $[0,500]$.\\
$\afm$ & Local human-mediated dispersal scaling factor & In the interval $[0,500]$.\\
$\ald$ & Long distance spread scaling factor & In the interval $[0,500]$.\\
\hline
\end{tabular}
\end{table}
%%
\paragraph{Network construction.} 
Figure~\ref{fig:pipeline} provides a schematic of the network construction.
The first step was to estimate monthly production volume of tomato,
eggplant and potato for each cell. We estimated annual production in each
cell followed by disaggregation to monthly production. The annual
production was estimated using the vegetable production available at the
highest resolution for each country (at the level of province to just one
value for the entire country) and a synthetic dataset called Spatial
Production Allocation Model. For monthly production, we
used linear regression to model the production rate as a function of
precipitation, temperature, and elevation. Seasonal tomato and eggplant
production data for different regions of Philippines was used. For most of
the other countries only qualitative information on seasonal of production
is available (Table~\ref{S:tab:countryData}). The regression function was
applied to locations of these countries where this information is
available and visually compared available data. The details are in
Section~\ref{S:sec:prod} of the supplement.

To model locality-to-locality trade, we applied the approach of
Venkatramanan~et~al.~\cite{venkatramanan2019modeling}
with some modifications. We modelled the flow of fresh tomato crop between
markets based on the following assumptions: (i)~The total outflow from a
city depends on the amount of produce in its surrounding regions and
imports from countries outside the focus region at time~$t$, and (ii)~the
total inflow depends on total consumption, processing demand, and exports
from the city to countries outside the focus region. The details are in
Section~\ref{S:sec:netcon}.  Trade between countries of the focus region
was not modelled as there is no adequate information on ports of entry or
monthly flow volumes.

\paragraph{Parameterization and experiment design.}
The goodness of fit of a parameter instance was determined by comparing the
simulation output with \tuta{} incidence reports
(Figure~\ref{fig:bgdClassA} and Table~\ref{S:tab:bgdData} for Bangladesh). The
spread was simulated with infestation starting from the location of first
report. For each cell, empirical probability that
it is in state~$I$ at time~$t$ was computed (averaged over 100
repetitions). The output was compared with ground truth using a similarity
function adapted from~\cite{carrasco2010unveiling}.  Let~$v$ be a reporting
cell and~$t_v$ denote the month of actual report of pest presence.  To
account for uncertainty in reporting, we consider a time
window~$U_\tau=[t_v-\tau,t_v+\tau]$ during comparison, where~$\tau$ is the
uncertainty parameter. We set~$\tau=2$, i.e., error within $\pm2$ months is
tolerated.  Supposing~$\reportingCells$ is the set of cells corresponding
to ground truth,  and ~$p(C,v,t)$ is the empirical probability that cell~$v$
is infected at time~$t$ for the input configuration~$C$, then, the similarity~$\similarity$ is
given by,
%%
\begin{linenomath}
\begin{align}\label{eqn:similarity}
    \similarity(C)=\frac{1}{|\reportingCells|}\sum_{v\in\reportingCells} \Big(\sum_{t\in U_\tau}p(C,v,t)
    + \sum_{t\notin U_\tau}\big(1-p(C,v,t)\big) \Big)\,.
\end{align}
\end{linenomath}
%%
$\similarity(C)$ attains a maximum value of one when simulation output is a
perfect match (within the error tolerance limit). Parameter space
exploration was conducted in multiple iterations (The range or values of
model parameters are in Table~\ref{tab:param}). First, we coarsely sampled
the space. With model parameters as independent variables and the goodness
of fit measure defined in~\eqref{eqn:similarity} as the dependent variable,
we used Classification and Regression Trees (CART) approach to identify
subspaces for which similarity score was high and rejected subspaces for
which similarity was low. Based on this, in the subsequent phases, we
refined our search to improve the parameterization. Simulations were
performed on more than~$500,000$ parameter combinations using a high
performance computing cluster. Configurations with high similarity score
($\similarity(C)\ge0.75$) were chosen for further analysis.

\paragraph{Analysis of spread pattern.} The objective here is to analyse
the variability in the simulation outcomes within the set of best fit
configurations. We leverage well-known machine learning techniques in a
novel way to address this question. The methodology is 
outlined in Figure~\ref{fig:clusterOutline}. First, we clustered
the simulation outputs (time and cell indexed empirical probabilities) of
selected configurations from the parameterization phase. This step captures
the variability in outcomes; simulation outputs belonging to different
clusters can be considered to be significantly different from one another.
In the second step, we attempt to unearth relationships between model
parameters and cluster membership. Our approach was to cast this as a
classification problem using CART with model parameters as the features
and cluster index as the label. The relationships
were inferred from the decision tree that results from the algorithm.
To avoid any bias introduced by the clustering algorithm, we applied more
than one method -- hierarchical agglomerative clustering and
$k$-means algorithm. In both cases, we used the Euclidean
distance as the distance measure to compare two simulation outputs. The
analysis was repeated for different values of~$k$, the number of clusters.
More details are provided in Section~\ref{S:sec:cluster} of the supplement.
%%
\begin{figure}[htb]
    \centering
    \includegraphics[width=.8\textwidth]{figs/spread_analysis.pdf}
    \caption{Outline of the process of analysing the multi-pathway spread. \label{fig:clusterOutline}}
\end{figure}
%%
\section{Results}
\paragraph{Variability in spread pattern.} The clustering analysis of the
configurations selected during the parameterization phase (approximately
8000 of them) reveals two distinct spread patterns primarily determined by
the pathway parameters.  The first class of models
(Figure~\ref{fig:bgdClassA}), referred to as Class~A, are characterised by
the absence of long-distance human-mediated spread ($\ald$ negligible) and
brisk spread between geographically adjacent cells, driven by latency
period~$\ell$ Moore range~$\mooreRange$, and short-distance scaling
factor~$\asd$. In contrast, in Class~B models
(Figure~\ref{fig:bgdClassB1}), the long-distance pathway~($\ald$) plays a
significant role  and there is relatively slow spread between
geographically adjacent neighbours. Both hierarchical clustering and
$k$-means clustering (Figures~\ref{S:fig:cartAgglomerative}(b)
and~\ref{S:fig:cartkmeans}(a)) are consistent in this regard.

%%
\begin{figure}[t]
    \centering
\begin{subfigure}[b]{.3\textwidth}
    \includegraphics[width=\textwidth]{../cellular_automata/results/contour/BGD_model-A.pdf}
    \caption{Class~A ($\ald=0$) \label{fig:bgdClassA}}
\end{subfigure}\hspace{.5cm}
\begin{subfigure}[b]{.3\textwidth}
    \includegraphics[width=\textwidth]{../cellular_automata/results/contour/BGD_model-B_m1_l3.pdf}
    \caption{Class~B ($\ald>0$) \label{fig:bgdClassB1}}
\end{subfigure}
\begin{subfigure}[b]{.32\textwidth}
    \centering
    \includegraphics[width=0.9\textwidth]{../cellular_automata/results/rf/rf_importance_all_mdi.pdf}
    \includegraphics[width=0.9\textwidth]{../clustering/results/agglomerative/rf_k_agglomerative_mse.pdf}
    \caption{Parameter importance \label{fig:rf}}
\end{subfigure}
\caption{\textbf{Spread in Bangladesh.} The contour plots show the simulated
spread starting from the location of first report in Panchagarh district
for 12 months. For the purpose of plotting, the time of infection for a
cell is the minimum time step~$t$ such that the empirical probability that
the cell is infected by time~$t$ is $\ge0.8$. Also highlighted are the
eight monitored locations and the localities applied in the model. The
colours of the monitored locations correspond to the month of report
relative to the first report (Panchagarh). Two distinct spread patterns
emerged from the cluster analysis. (a) and (b) show representative spreads
observed for each class. The similarity ($\similarity$) in each case was
$>6.5$. Importance of model parameters with respect to (i)~similarity
score~$\similarity$ and (ii)~cluster membership based on random forest
method. The latter plot shows how the results vary with increase in number
of clusters for hierarchical clustering algorithm. More results are
presented in Figure~\ref{S:fig:rf} in the supplement.}
\end{figure}
%%

Class~A spread pattern does not capture the gap between the time of first
report (Panchagarh) and the report in Gaibandha district
(Figure~\ref{fig:bgdClassA}). Even though the distance between the two
locations is only $185$km, the latter reported the presence only after~10
months of first report suggesting that self-mediated spread might have been
much slower. In the model output on the other hand, the corresponding cell
gets infected between the second and fourth months.  In Class~B, this
location is infected much later in comparison. However, the eastward spread
towards the location Jaintiapur is slower than what was observed
(Figure~\ref{fig:bgdClassB1}). Even though Panchagarh is quite far from
this location, pest presence was reported by February 2017, just nine
months from the first report.  As a baseline, we also simulated the spread
using the cellular automata model developed by
Guimapi~\cite{guimapi2016modeling} for Bangladesh. The spread pattern is
similar to Class~A as the model does not account for long-distance hops.
However, the predicted rate of range expansion is much higher than our
models (see Section~\ref{S:sec:guimapi} for model details and results).

The relative importance of model parameters was assessed using random
forest algorithm with regard to their influence on (i)~similarity score
similarity score ($\similarity$) and (ii)~spread pattern, which in our
case, is akin to cluster membership. In the case of spread pattern, the
importance was derived for each~$k$ (number of clusters) and clustering
algorithm. Some results are presented in Figure~\ref{fig:rf}. We note that
the long distance scaling factor ($\ald$) is among the top three important
parameters. The start month ($t_s$) is also important due to two reasons.
Firstly, the distance between two time shifted simulation outputs can be
large. Secondly, outputs are sensitive to seasonal variations or
temporality of the network.  Latency period ($\ell$) and Moore range
($\mooreRange$) together control the extent of radial spread in a time
step. Typically, for Class~A models, $\mooreRange$ is high and $\ell$ is
low and the other way round in the case of Class~B models. Analysis of
trade flows and seasonality is presented in
Section~\ref{S:sec:locality_flows}
in the supplement.

\paragraph{Scenarios of pest introduction to countries in Southeast Asia.}
To identify routes of introduction to other countries in the region, we
applied both Class~A and Class~B models. The starting point of the spread
corresponds to Panchagarh district (Figure~\ref{fig:bgdClassA}). Both model
classes strongly indicate that \tuta{} is already present in parts of
Myanmar (curves corresponding to time step 24 or two years from first
report). Also, the pest is likely to enter Thailand from Myanmar, and
subsequently move to Laos and Vietnam as it spreads eastwards and to China
when it spreads northwards. From Thailand, spreading southward, it will
enter Malaysia and subsequently enter Indonesia
(Figure~\ref{fig:msaClassAB}).

%%
\begin{figure}[ht]
    \centering
\begin{subfigure}[b]{.47\textwidth}
    \includegraphics[width=\textwidth]{../cellular_automata/results/contour/MSA_model-A_m2_l1.pdf}
    \caption{Class~A\label{fig:msaClassA}}
\end{subfigure}\hspace{.5cm}
\begin{subfigure}[b]{.47\textwidth}
    \includegraphics[width=\textwidth]{../cellular_automata/results/contour/MSA_model-B_m1_l3.pdf}
    \caption{Class~B\label{fig:msaClassB}}
\end{subfigure}
\caption{\textbf{Spread pattern in Mainland Southeast Asia without
accounting for trade between countries.} The contour plots show the
simulated spread starting from northern Myanmar for 120 time steps or 10
years. Representative simulation outputs for Class~A and Class~B models are
shown. With {Class~A}, spread is faster eastward than southward and the
other way round in the case of Class~B.\label{fig:msaClassAB} }
\end{figure}
%% %%   

We also analysed the international tomato
trade network (Section~\ref{S:sec:tomnet} of supplement) to assess the risk
due to imports from \tuta{} infested countries outside this region.
Malaysia and Singapore are important hubs with
tomato imports from \tuta{} infested regions. There is a possibility that
\tuta{} is directly introduced to these regions.  However, in both cases,
the import volume is very low. Also, the introduction risk depends on the
preventive measure taken by the exporting countries. With respect to both
trade and natural pathway, there is a low chance that the pest will be
introduced into Philippines from neighbouring countries.  This is because,
it does not share its borders with any country in the region and also,
there is no evidence of tomato trade with rest of the countries. However,
human mobility is a possible pathway. The Middle East is the top
destination for Filipino workers.%%~\cite{rodriguez2011philippine}.
%%
\paragraph{Predicted spread is model and region dependent.} In the case of Class~A
models, the eastward spread is faster than southward spread
(Figure~\ref{fig:msaClassA}). This is mainly because the Moore neighbourhood
is smaller at the narrow region in the south of Myanmar and Thailand
bordering Malaysia.  However, in the case of Class~B
(Figure~\ref{fig:msaClassB}), the spread is much faster in the same region
aided by domestic trade flows from northern and central Thailand to the
southern region.  Class~A spread pattern predicts that within the next 4-5
years much of the northern part of Mainland Southeast Asia will be invaded.
Class~B spread pattern predicts that in the same period \tuta{} will spread
all over Malaysia and Singapore.  However, the rate of spread observed is
slower than that observed in Bangladesh for both classes. Also, even though
the models exhibited similar rate of spread for Bangladesh, we observed
high variance in intensity of infestation as well range expansion for the
rest of the region. The results are in Figure~\ref{S:fig:spreadRate}. The
reason for slow spread is as follows.  Bangladesh has the highest tomato
volume per country surface area ($\approx2.5$tonnes/km$^2$).  The next
country is Vietnam ($\approx1.5$tonnes/km$^2$). Therefore, in the case of
Bangladesh, not only is the extent of infestation in a
cell~$\infest(\cdot)$ typically high, but also, since it is a densely
populated country, most cells have vegetable production. Hence, the rate of
spread is much higher for relatively lower values of pathway parameters and
Moore range. Also, we observed a strong dependence on Moore range
(Figure~\ref{S:fig:spreadRateB}).  In countries with larger area, the
production is scattered. Therefore, lower the Moore range, the slower the
spread.

\paragraph{Influence of domestic trade on spread pattern and rate.}
Here, we focus on long-distance dispersal, and therefore, restrict our
discussion to Class~B models. For the country-specific studies, the
starting location was decided based on our analysis of possible entry
points through different pathways (Section~\ref{S:sec:seeds} in the
supplement).  We observed the following common spread pattern.  When the
invasive species is introduced to a country, dispersal is slow until the
invasion front reaches a {production source}. Once it establishes at a
source, the spread is very fast.  Depending on the country, within~12 to~24
time steps (or an year or two), it spreads to almost all major localities of
the country (see Figure~\ref{fig:thlBContourInt} for example). Production
areas which are very close to high-consumption localities (large urban
areas) are particularly vulnerable. Since local production typically does
not satisfy demands of such localities, they have high inflows from other
production areas and possibly from other countries. As a result, these
localities are the quickly infected. Once introduced to such localities,
farmer--market interactions (local human-mediated dispersal) can facilitate
the introduction of the pest to nearby production regions where it can
establish and proliferate.

%%
Given that monitoring and quarantining are both resource intensive and
potentially disruptive, developing strategies that involve few locations,
yet provide near-optimal control is a goal for modellers.  Market-level
phytosanitary measures in terms of import restrictions have been undertaken
by countries~\cite{USDA2012}. Here, we evaluated a simple strategy of
containing the spread through the trade pathway. Localities associated with
high annual outflows were identified (at most four in each country). As
discussed earlier, pest establishment in these areas can potentially lead
to rapid range expansion. The outflow from the targeted localities was cut
off to mimic control at the trade/market level. In the strictest sense,
this can be implemented by restricting trade of host crops. But, it is
possible that phytosanitary measures have the same effect.
Figure~\ref{fig:spread} shows results for two countries.  More results are
present in Figure~\ref{S:fig:intervene} in the supplement.  Consistently,
across countries, we observed a significant reduction in range expansion as
well as intensity of spread.  Besides, as seen in
Figure~\ref{fig:thlBContourInt}, stifling these flows localises the spread
that resembles those of Class~A models, but with much less intensity.
%%
\begin{figure}[ht]
\begin{subfigure}[b]{.28\textwidth}
\includegraphics[width=\textwidth]{../cellular_automata/results/contour/TH_model-B_precip1_m1_l3.pdf}
\caption{Thailand without intervention\label{fig:thlBContour}}
\end{subfigure}
\begin{subfigure}[b]{.28\textwidth}
\includegraphics[width=\textwidth]{../cellular_automata/results/contour/TH_model-B_precip1-out-100_m1_l3.pdf}
\caption{Thailand with intervention\label{fig:thlBContourInt}}
\end{subfigure}
\begin{subfigure}[b]{.43\textwidth}
\includegraphics[width=\textwidth]{../cellular_automata/results/dist_inf_plots/TH_dist_prob_B_box.pdf}
\caption{Thailand: range of expansion (all Class B models)\label{fig:thlBContourBox}}
\end{subfigure}
\begin{subfigure}[b]{.28\textwidth}
\includegraphics[width=\textwidth]{../cellular_automata/results/contour/VN_model-B_precip1_m1_l3.pdf}
\caption{Vietnam: without intervention\label{fig:vnmBContour}}
\end{subfigure}
\begin{subfigure}[b]{.28\textwidth}
\includegraphics[width=\textwidth]{../cellular_automata/results/contour/VN_model-B_precip1-out-100_m1_l3.pdf}
\caption{Vietnam: with intervention\label{fig:vnmBContourInt}}
\end{subfigure}
\begin{subfigure}[b]{.43\textwidth}
\includegraphics[width=\textwidth]{../cellular_automata/results/dist_inf_plots/VN_dist_prob_B_box.pdf}
\caption{Vietnam: range of expansion (all Class B models)\label{fig:vnmBContourBox}}
\end{subfigure}
\caption{\textbf{Rate and pattern of spread with and without intervention.}
Representative spread dynamics of Class~B models ($\mooreRange=1, \ell=3$)
for two countries. More plots are in Figure~\ref{S:fig:intervene}. In each
case, a cell close to a high production region was seeded. The first column
corresponds to spread for 48 months after introduction. The colours indicate
the time interval at which there is at least a 50\% chance that a location
will be infected.The second column corresponds to spread after cutting off
flows from chosen localities. The third column shows average spread with
respect to origin of infection for all Class~B models. The cells are binned
based on their distance from the origin of infection. Given time step~$t$
(48), let $\Pr(v,\le t)$ be the probability that cell~$v$ is in state~$I$
by time~$t$. For each configuration, we computed the ``total
infection'' for every bin at time~$t$ by aggregating~$\Pr(v,\le t)$ for
each~$v$ in the bin. The red points referred to as ``max'' correspond to
the total number of cells in each bin, which is also the maximum possible
accumulated probability for that bin.
\label{fig:spread}}
\end{figure}

%%
\section{Discussion}
The variability in the spread patterns that explain the incidence reports
exposes the lack of understanding of the pathways of spread.  Nevertheless,
the analysis does strongly indicate the role of human-assisted spread of
\tuta{}. The pest was reported in May~2016 in the northwestern part of
Bangladesh bordering India. The region is among the top three tomato
producers in the country. By the beginning of the next production season,
\tuta{} was found in almost every major urban region.  Similar correlation
between tomato trade and \tuta{} spread was observed in
Nepal~\cite{venkatramanan2019modeling}. Studies on self-mediated spread
(flying capability or by wind) can definitely help estimate more accurately
the rate of self-mediated spread. It is also important to consider
alternate scenarios of introduction. We recall that the far eastern part of
Bangladesh (locality Jaintiapur in Figure~\ref{fig:bgdClassB1}) reported
pest presence nine months after the first report. In a typical Class~B
spread pattern, however, this region was not infected within that time
frame mainly because of week trade flows to that region. However, in
January 2017, \tuta{} was officially reported from Meghalaya in India,
about 100km from Jaintiapur. It also happens to be near an important trade
route from India to Northeastern Bangladesh. Therefore, it is possible that
multiple incursions took place. It is possible that analysis of other
countries or regions can reduce this variability. But, caution needs to be
exercised before applying analysis of one region to another as trade
dynamics, production methods and as a result pathways can wary widely from
one region to another.

We recall the discussion on slow predicted rate of spread in Mainland
Southeast Asia compared to the observed rate in Bangladesh. One reason for
this could be the unaccounted trade flows between countries. International
trade within this region is not documented well. Historically,
international trade has played a strong role in the spread of \tuta{}
between countries. For example, the pest was first reported by India
in~2014. By early~2016, it was discovered in the Kathmandu area of Nepal,
the northern part of Bangladesh in May~2016. Both countries import
significant volume of tomato from India. However, there has been no report
from Pakistan, another neighbour which does not import tomato from India. It
is possible that the pest is present and undetected in this country. But,
it is clear that Pakistan is aware of the threat\footnote{The looming
threat of the deadly tomato
leafminer~(\url{https://www.dawn.com/news/1420206}).} indicating that even
if it is present, it is not widespread. There are similar examples outside
the region such as its slow advance from South America to Central America,
or the fact that it is not reported in China despite being present in
neighbouring Central Asian countries since~2015.  Hence, it is critical to
address the data gaps concerning international trade.

While, several integrated pest management strategies have been suggested for managing \tuta{},
hardly any work has been done in designing effective interventions at the
trade level. Designing phytosanitary measures targeted towards market and
vehicles of transportation for preventing introductions (or
reintroductions) is therefore a promising research direction. Some
countries have already taken measures in this regard. Our results also
indicate that monitoring markets with inflows from many regions is
important. In the United States for example, the Animal and Plant Health
Inspection Service of the Department of Agriculture (USDA-APHIS) has
instituted quarantine regulations for imports from regions where the pest
is present~\cite{USDA2012}. Identifying the optimal set of nodes in a
network to reduce infectious disease spread is a widely studied topic.
There are very few works that apply such techniques to invasive species
spread~(Nopsa~et~al.~\cite{nopsa2015ecological} for example). As the world
moves towards concentrated and specialised agricultural production,
focusing on this aspect becomes increasingly important.
%%~\cite{madar2004immunization}
%%
\paragraph{Literature survey.} Multi-pathway models are being increasingly
used to study the role of invasive species dispersal.
Douma~et~al.~\cite{douma2016pathway} survey the literature categorising
various efforts into flow-based pathway models and agent-based models.
Carrasco~et~al.~\cite{carrasco2010unveiling} combine spatially
explicit models of human-mediated spread with a phenology model to
incorporate population dynamics of the western corn rootworm.
Nopsa~et~al.~\cite{nopsa2015ecological} use a network science approach to
studying the role of transport and storage infrastructure in the spread of
pests and pathogens of wheat. Our model is in part motivated by the hybrid
approaches used in the study of infectious diseases of humans and
livestock~\cite{bradhurst2015hybrid,yang2016}.
Although there is a general consensus that vegetable and seedling trade is
a primary driver of \tuta{} spread, previous modelling efforts have
exclusively focused on ecological aspects. Several
studies~\cite{desneux2010biological,tonnang2015identification} provide risk
maps using CLIMEX and take additional factors into account.
Guimapi~et~al.~\cite{guimapi2016modeling} used a cellular automata approach
to capture the global spread of the pest by factoring in temporal
variations and spatial distribution of vegetation, temperature, and tomato
production. A precursor to this work~\cite{venkatramanan2019modeling}
modelled the seasonal production and trade of tomato in Nepal to study the
role of trade in the spread of \tuta{} in Nepal using gravity model and
network dynamics.

Emulators --based on Gaussian processes for
example~\cite{fadikar2018calibrating} -- and machine learning
surrogates~\cite{lamperti2018agent} are emerging as solutions to overcome
computational challenges, parameterization and sensitivity analysis of
complex agent-based models. The parameter importance study and
parameterization was partly motivated by the work of
Lamperti~et~al.~\cite{lamperti2018agent}. We are not aware of any previous
work that analyses the dynamics of simulation systems using unsupervised
learning presented in this paper. However, clustering has been considered
in the context of multi-resolution simulation models as an interfacing
component between simulators with different
resolutions~\cite{cassandras2000clustering}. We cast the problem of
deriving relationship between model parameters and cluster index as a
classification problem. CART was our choice of algorithm since the
learned model is a decision tree that can be interpreted. In principle, any
such algorithm which provides such as explanation (such as multinomial
regression for example) can be employed.

%%
\paragraph{Challenges and limitations.} Modelling emerging invasions is
particularly challenging. Limited data on incidence and understanding of
the underlying dynamics makes it nearly impossible to calibrate and
validate the models.   We have had to simplify or ignore some of the
processes that might significantly influence the spread.  For example, our
model uses monthly production as a surrogate for infectiousness of a cell.
Complex phenology models can be used instead (as in
Carrasco~et~al.~\cite{carrasco2010unveiling}), but would add to the
complexity of the model.
Since our focus region spans multiple countries, identifying and collecting
data for each country was a tedious process. For many countries, data had
to be collected (or even inferred) from several publications and reports
(Table~\ref{S:tab:countryData}). Further, these datasets were misaligned in
time and spatial resolution.  It is important to account for heterogeneity
in production, consumption, awareness, cultural factors, etc. both within
and between countries.  Some countries are technologically more advanced
than others, which manifests as differences in yield, crop loss, trade
infrastructure, pest awareness and preparation for
invasion~\cite{early2016global}.

In particular, it is hard to model human assisted spread owing to lack of
seasonal trade data. To
determine outflows and inflows for each locality, we had to identify major
ports for imports and exports as well as estimate fraction of production
which was used for processing, which was available only for a few
countries. The farm--market-consumer interactions (local human-mediated
spread) involves various actors such as farmers, wholesalers, retailers,
wet markets, supermarkets, etc. Modelling this is a challenge in itself. If
data on actual flow of vegetables is provided, the gravity model can be
improved or replaced by more sophisticated approaches. Also, the
relationship between long-distance invasion risk and trade volume is hard
to determine. While a direct relationship between volume and risk is
plausible, whether the relation is linear (as assumed by our model) is not
clear.

\paragraph{Conclusion.} Traditionally, in developing countries, crops such
as tomato are seasonal. However, over the past decade, due to rising demand and
opportunities to export, there has been a thrust towards year-round
production using protected cultivation methods and resilient varieties.
Increase in urban population, short shelf life of vegetables and the
advantages of short marketing chains have encouraged urban agriculture in
developing countries~\cite{moustier2015urban}. Our results indicate that
such urban and peri-urban agriculture is particularly vulnerable to
invasive species attacks. In particular, in Southeast Asia, vegetable
production and internal trade has steadily increased. In comparison, the
export of tomato to outside of the focus region has risen steeply in the
recent years (after 2011), while the imports generally indicate a downward
trend.  Therefore, invasions from pests such as \tuta{} can have a huge
negative impact on the socioeconomic fabric of this region. The modelling
and analysis framework presented here is generic and applicable to other
invasive species.  The methodology is modular and leverages popular
learning algorithms to analyse complex models under data scarcity.  Other
potential applications for this work include studies of natural or
human-initiated disasters, climate change and optimisation of food flows.
\paragraph{Data availability.} The authors declare that the data supporting the
findings of this study are available within the paper and its Supplementary
Information file, or from the authors upon reasonable request.

\paragraph{Acknowledgements}
This work was supported in part by the United States Agency for
International Development under the Cooperative Agreement NO.
AID-OAA-L-15-00001 Feed the Future Innovation Lab for Integrated Pest
Management, DTRA CNIMS Contract HDTRA1-11-D-0016-0001, NSF BIG DATA Grant
IIS-1633028, NSF DIBBS Grant ACI-1443054, NIH Grant 1R01GM109718 and NSF
NRT-DESE Grant DGE-154362.  We are grateful to Yousuf Mian, Nguyen Van Hoa,
and Kimhian Seng for their help with obtaining country-specific information
on production, trade, and pest incidence. We thank Richard Beckman and
Irene Eckstrand for useful discussions on model design and paper
organisation.

\paragraph{Author contributions.}
AA defined the scope of the
research. AA, JM, TB, MRC collected and interpreted data.
AA, MM conceived and designed the
experiments. JM, AA and YYC performed the
analysis. HM, ND, TB and RM provided assistance in interpreting the
results. AA and JM wrote the paper with significant inputs from
MRC and YYC. AA supervised the research. All authors discussed the
results and commented on the manuscript.

\bibliographystyle{abbrv}
\bibliography{refs}
%%
\end{document}

\PassOptionsToPackage{table}{xcolor}
\documentclass[11pt]{article}
%
%%%%%%%%%%%%%%%%%%%%%%%%%%%%%%%%%%%%
%% Common preamble
%%%%%%%%%%%%%%%%%%%%%%%%%%%%%%%%%%%%
% PAGE
\newcommand{\tuta}{\emph{T.~absoluta}}
\newcommand{\infest}{\rho}
\newcommand{\totinf}{\rho_{\mathrm{T}}}
\newcommand{\suitable}{\epsilon}
%% \usepackage{fullpage} % uncomment this when changing to jour/conf style
% FONTS
\usepackage{textcomp}
\usepackage{lmodern} % enhanced version of computer modern
\usepackage[T1]{fontenc} % for hyphenated characters and textsc in section title
\usepackage{xr}
\externaldocument[S:]{supplementary}
\usepackage{microtype} % some compression
\usepackage{times}
\usepackage{setspace} % For double line spacing
\doublespacing
\usepackage[left=1in,top=1in,right=1in,bottom=1in]{geometry}
\usepackage[pagewise]{lineno}
\linenumbers
\usepackage{marvosym}
%% \newcommand\wordcount{\immediate\write18{texcount -sub=section \jobname.tex  | grep ``Section'' | sed -e 's/+.*//' | sed -n \thesection p > 'count.txt'} (\input{count.txt}words)}
% MATH
\usepackage{amssymb}
\usepackage{mathtools} % contains amsmath which comes with align
\usepackage{amsthm} % newtheorem stuff
\usepackage{bm} % for bold math (use $\boldsymbol{}$)
% COLOR
\usepackage[usenames,dvipsnames]{color}
%%
% REFERENCING
% JM: Below is for affiliations
\usepackage{authblk}
% BIBLIOGRAPHY
\usepackage[square,sort&compress,numbers]{natbib} %sorts bibs when they are collectively cited
\usepackage[colorlinks=true,pdfborder={0 0 0},citecolor=RoyalBlue,linkcolor=black,urlcolor=Magenta]{hyperref}
\usepackage{cleveref} %%% IMPORTANT: cleveref comes before \newtheorem commands
   \crefname{figure}{Figure}{Figures}
   \crefname{table}{Table}{Tables}
   \crefname{theorem}{Theorem}{Theorems}
   \crefname{lemma}{Lemma}{Lemmas}
   \crefname{claim}{Claim}{Claims}
   \crefname{section}{Section}{Sections}
   \crefname{observation}{Observation}{Observations}
   \crefname{note}{Note}{Notes}
%%
% TABLES
\usepackage{multirow}
%% \usepackage{ctable} % provides toprule, bottomrule, midrule
\usepackage{array} % new implementation of tabular & array with lots of enhancements
%%
% FIGURES
\usepackage{graphicx}
\graphicspath{./figs}
\usepackage{grffile} % to set right names of files
%%
% CAPTIONS
\usepackage{caption}
\usepackage{subcaption} % supersedes subfigure & subfloat. try using options
%%
% LISTS
\usepackage{enumitem} %\begin{itemize}[leftmargin=*]
%% inline
\newlist{inline}{enumerate*}{1}
\setlist[inline]{before=\unskip{: }, itemjoin={{; }}, itemjoin*={{; and }}, label={(\roman*)}}
\usepackage{silence}
\WarningFilter{ctable}{Transparency disabled:}
\WarningFilter{xcolor}{Incompatible color}
% ALGO
\usepackage[ruled,linesnumbered]{algorithm2e}
%% comments & todo
\newcommand{\reportingCells}{\mathcal{C}_R}
\newcommand{\similarity}{\mathcal{S}}
\newcommand{\aacomment}[1]{({\color{magenta}AA: #1})}
\newcommand{\mmcomment}[1]{({\color{green}MM: #1})}
\newcommand{\tbcomment}[1]{({\color{blue}TB: #1})}
\newcommand{\mrccomment}[1]{({\color{red}MRC: #1})}
\newcommand{\jmcomment}[1]{({\color{cyan}JM: #1})}
\usepackage[colorinlistoftodos]{todonotes}
\newcommand{\TODO}[1]{\todo[inline,color=red!10,size=\small]{#1}}
%% COMMANDS
\newcommand{\comp}[1]{\overline{#1}}  %{\widetilde{#1}}
\DeclareMathOperator{\Var}{Var}
\DeclareMathOperator{\Cov}{Cov}
\DeclareMathOperator{\bigo}{O}
\DeclareMathOperator{\bigom}{\Omega}
\DeclareMathOperator{\davg}{d_{avg}}
\DeclareMathOperator*{\argmax}{arg\,max}
\DeclareMathOperator*{\argmim}{arg\,min}
% Note: \deg is already defined
\newcommand{\expect}{\mathbb{E}}
%
\newcommand{\ceil}[1]{\left\lceil #1 \right\rceil}
\newcommand{\floor}[1]{\left\lfloor #1 \right\rfloor}
%
\newcommand{\reals}{\mathbb{R}}
\newcommand{\field}{\mathbb{F}}
\newcommand{\integers}{\mathbb{Z}}
\newcommand{\pshort}{p_s}
\newcommand{\plocal}{p_{\ell}}
\newcommand{\pld}{p_{\ell d}}
\newcommand{\asd}{\alpha_s}
\newcommand{\afm}{\alpha_{\ell}}
\newcommand{\ald}{\alpha_{\ell d}}
\newcommand{\produce}{\mathrm{Prod}}
\newcommand{\veg}{\mathrm{V}}
\newcommand{\temp}{\mathrm{T}}
\newcommand{\consume}{\mathrm{Pop}}
\newcommand{\locality}{\mathrm{L}}
\newcommand{\export}{\mathrm{Export}}
\newcommand{\import}{\mathrm{Import}}
\newcommand{\process}{\mathrm{Proc}}
\newcommand{\moore}{\mathrm{M}}
\newcommand{\mooreRange}{r_\mathrm{M}}
%
\newtheorem{theorem}{Theorem}[]{\bfseries}{\itshape} 
\newtheorem{lemma}[theorem]{Lemma}{\bfseries}{\itshape}
\newtheorem{claim}[theorem]{Claim}{\bfseries}{\itshape}
\theoremstyle{definition}
\newtheorem{definition}[theorem]{Definition} % {\bfseries}{\itshape}
\newtheorem{observation}[theorem]{Observation} % {\bfseries}
\newtheorem{condition}[theorem]{Condition} % {\bfseries}{\itshape}
\newtheorem{note}[theorem]{Note} % {\bfseries}{\itshape}
% ROMAN NUMERALS
\makeatletter
\newcommand{\rmnum}[1]{\romannumeral #1}
\newcommand{\Rmnum}[1]{\expandafter\@slowromancap\romannumeral #1@}
\makeatother
%%
% TWO VERSIONS
%% \usepackage{etoolbox}
%% \newtoggle{withappendix}
%% \toggletrue{withappendix} % comment this if you want journal version
%% Usage: iftoggle{withappendix}{}{}
%%
%% math operator
%% \DeclareMathOperator{\sgn}{sgn}
% CODEBOX
%% \usepackage[framemethod=tikz]{mdframed}
%% \newmdenv[linecolor=black!10,innerlinewidth=0pt, roundcorner=4pt,innerleftmargin=6pt,
%% font=\ttfamily,innerrightmargin=6pt,innertopmargin=6pt,
%% innerbottommargin=6pt,backgroundcolor=black!10]{codeblock}
%%%%%%%%%%%%%%%%%%%%%%%%%%%%%%%%%%%%
%% preamble ends
%% from now on, draft specific
%%%%%%%%%%%%%%%%%%%%%%%%%%%%%%%%%%%%
%% \RequirePackage[l2tabu, orthodox]{nag}
\makeatletter
\renewcommand\AB@affilsepx{, \protect\Affilfont}
\makeatother
% Title on edge of suggested length
\title{Assessing the Multi-pathway Threat from an Invasive Agricultural
Pests} %: \emph{Tuta~absoluta} in Asia}
%% \title{A Multi-pathway Model to Assess the Threat of Invasive Agricultural
%% Pests: Case Study of \emph{Tuta~absoluta} in Asia}
\author[1]{Joseph~McNitt}
\author[2]{Young~Yun~Chungbaek}
\author[2]{Henning~Mortveit}
\author[2]{Madhav~Marathe}
\author[3]{Mateus~Ribeiro~de~Campos}
\author[3]{Nicolas~Desneux}
\author[4,5,6]{Thierry~Br\'{e}vault}
\author[7]{Rangaswamy Muniappan}
\author[2]{Abhijin~Adiga}
\affil[1]{Department of Mathematics, Virginia Tech}
\affil[2]{Biocomplexity Institute \& Initiative, University of Virginia}
\affil[3]{French National Institute for Agricultural Research}
\affil[4]{BIOPASS, CIRAD-IRD-ISRA-UCAD, Dakar, Senegal}
\affil[5]{CIRAD, UPR AIDA, F-34398 Montpellier, France}
\affil[6]{Universit\'{e} de Montpellier, CIRAD, Montpellier, France}
\affil[7]{Feed the Future Integrated Pest Management Innovation Lab}
\date{}
\setcounter{Maxaffil}{0}
\renewcommand\Affilfont{\itshape\small}

\begin{document}
\maketitle

\begin{abstract}
Modern food systems facilitate rapid dispersal of pests and pathogens
through multiple pathways. Complexity of the spread dynamics and data
inadequacy make it hard to understand the phenomenon and better 
prepare for emerging invasions. We present a generic multi-scale modeling
framework to study the spatio-temporal spread of invasive species as a
propagation process over a time-varying network accounting for biology,
seasonal production, trade and demographic information. Model variability
and parameter sensitivity are analyzed using novel machine learning
techniques. We applied it to study the spread of a devastating pest of
tomato, \emph{Tuta absoluta}, in South and Southeast Asia -- a region at the
frontier of its current range. Analysis with respect to historical invasion
records suggests that even with modest self-mediated spread capabilities,
the pest can quickly expand its range through domestic city-to-city
vegetable trade. There is a strong chance that within five to seven years
\tuta{} will invade all major vegetable growing areas of Mainland Southeast
Asia if no steps are taken to mitigate the spread.  Further, we showed that
monitoring and effective interventions in major production areas could
greatly reduce the speed of the spread.
\end{abstract}
%%
\section{Introduction}
Intensification of trade and human mobility has led to an increase in
exotic species invasions~\cite{hulme2009trade}. Climate change and
detrimental impact of intensive agriculture on natural resources are likely
to further aggravate this
problem~\cite{early2016global,garrett2013agricultural}. Models of invasive
species spread play an important role in predicting the spatio-temporal
spread under different scenarios of introduction. They can also help
identify roles of different pathways, efficacy of control strategies and
expose gaps in the understanding of the
phenomenon~\cite{cunniffe2016modeling,epstein2008model}. However, impending
invasions of agricultural pests present difficult challenges to overcome.
Accounting for multiple drivers of dispersal invariably makes the model
complex and demands multi-disciplinary research. At the same time, data
inadequacy makes it nearly impossible to calibrate and validate these
models. Few and often inaccurate incidence reports, lack of production,
trade data, heterogeneity (multiple sources, different formats and
resolution) are some examples. A natural goal for a modeler is to develop
analytical tools that account for the complexity of spread dynamics, cope
with scarcity and variety of data, and provide useful insights for
designing effective policies to prevent and mitigate invasive species
spread.

talk about multi-scale. prevent overfitting
Given the complexity of the model, few data points and limited
understanding of the pest's behavior, it is possible that multiple spread
patterns explain the ground observations. Two natural questions arise: How
do we discover this variability (if any) in spread pattern? How do we
explain this variability in terms of model parameter regimes? 
Network dynamical systems have been effectively used to study large
interacting biological and social systems~\cite{}. In
some fields such as computational epidemiology~\cite{eubank2004modelling},
online social networks~\cite{guille2013information} and
transportation~\cite{transims}, they occupy a place among mainstream
modeling approaches.  These models are explicit algorithmic descriptions of
composed local interactions. The resulting dynamics of such a validated
model yields a causal description of the underlying complex system. A
potential challenge however is the large number of parameters used to
describe the system, particularly under data scarcity and model
uncertainty.  We propose novel ways to analyze the in which machine
learning algorithms can be applied to address these problems to increase
the utility of such models.  Here, we focus on the problem of invasive
species spread, in particular, pests and pathogens of important
agricultural crops.  Increasingly, network diffusion based models are being
applied to study such
phenomena~\cite{carrasco2010unveiling,nopsa2015ecological}.

see \cite{jordan2019modeling}


make sure intro has challenges

model, data uncertainty, pathway models, talk about Douma et al.

%%  which has invaded more than 60\% of
%% tomato producing land over the last decade.

Pest risk maps have evolved ... long-term establishment. 
 We developed a 

%% The world is witnessing a rapid increase in trade and
%% travel~\cite{ercsey2012complexity}. Due to this increased connectivity,
%% both international and domestic, no region is spared of the threat from
%% exotic species invasion~\cite{hulme2009trade}.   As a result, global food
%% security, human health and social welfare will be adversely impacted.

The South American Tomato leafminer, or \emph{Tuta absoluta}, a pest of the
tomato crop, is a representative among the recent biological invasions that
have significantly perturbed global food production. Indigenous to South
America, \tuta{} was accidentally introduced to Spain in~2006, and since
then it has rapidly spread throughout Europe, Africa, Western and Central
Asia, the Indian subcontinent, and parts of Central
America~\cite{desneux2010biological,biondi2017}. With tomato being a
commercially important crop, this invasion has had significant global
impact~\cite{campos2017western}. In the Netherlands and Turkey alone, the
annual estimated intervention cost are \EURdig4 million and \EURdig167
million per year~\cite{}, respectively. It is well accepted that trade
played a critical role in \tuta{}'s rapid spread. On multiple occasions it
has been discovered in packaging stations and its spread pattern is
correlated with prime trade routes~\cite{karadjova2013tuta}. 

%% The wide-spread
%% adoption  of green houses and tunnel farming have allowed it to overwinter
%% as well as survive the wet season. 
%% There is a great need to understand the
%% dynamics of this pest in order to predict its future invasion pattern, and
%% design effective prevention and mitigation policies.

%% Due to extensive insecticide
%% treatment in Europe, insecticidal resistance has been recently observed in
%% populations~\cite{guedes2013tomato}. 
%% Overall,
%% lack of effective indigenous predators has made integrated pest management
%% (IPM) a challenging task.
%%~\cite{potting2013tuta,oztemiz2014tuta}

%% Since tomato is among the top two traded vegetables in the
%% world\footnote{\url{http://www.fao.org}}, 

%% Faced with the challenging task of preparing for the invasion of pests and
%% pathogens, and responding effectively to mitigate such incursions should
%% they happen, decision makers are increasingly relying on  computational
%% models of pest risk maps and spread to aid decision
%% making~\cite{venette2010}.  While models that create pest risk maps (e.g.
%% CLIMEX) are useful to identify locations which are suitable for long-term
%% establishment, % and cause harmful impact spread models help specify the
%% spatio-temporal dynamics of how the pest spreads.  The latter approach is
%% particularly useful in planning for the possible threat from an invasive
%% species~\cite{perrings2014merging,barbier2013implementing,paini2010threat,paini2010threat,paini2016global}
%% through {\it in silico} experiments simulating hypothetical invasion
%% scenarios. 
%% Together, these tools can be used to address questions such as
%% which locations to monitor, what control measures to take (areawide IPM
%% practices, trade restrictions, etc.), what is the impact on the economy and
%% health, and so on. 

We describe a multi-pathway propagation model 
to study the spread of \tuta{} in the region of South
and Southeast Asia comprising of 10 countries: members of the Association
of Southeast Asian Nations (ASEAN) and Bangladesh. To develop this model,
we identified, analyzed, and fused disparate datasets corresponding to
ecology,  
%% (temperature, precipitation, elevation, vegetation, host
%% plant presence), 
%% (host preference, suitability, population growth),
production, trade 
%% (imports, exports, domestic trade, city locations, city to city distances), 
and demographic factors. 
%% (per capita consumption and population). 
Learning methods such as Classification and Regression Trees
(CART)~\cite{breiman2017classification} and Random Forests were used for
parameter space exploration and sensitivity analysis.  To our knowledge,
this is the first study that explicitly considers multiple pathways of
introduction and spread of \tuta{}.  Earlier modeling efforts have only
accounted for ecological aspects and self-mediated
spread~\cite{desneux2010biological,tonnang2015identification,guimapi2016modeling}.
A major hurdle in modeling spread of emerging pests and pathogens in data
poor regions is the lack of accurate incidence reports, production and
trade dynamics. To fill these gaps we gathered quantitative and
qualitative information from more than 50 research articles and reports
that analyze vegetable production in the region.

%% This problem is compounded in data-poor regions due to
%% unavailability and inaccessibility of data on production and trade, making
%% parameterization and validation nearly impossible. 
%% For
%% parameterization and sensitivity analysis of this complex model, we used a
%% Classification and Regression Trees approach~\cite{lamperti2018agent}.


This is a timely and important study for several reasons. In the recent
years, there has been a thrust to improve vegetable production in all the
countries of this region~\cite{ali2001}. With the pest having
already spread to major tomato producing areas in
Bangladesh~\cite{hossain2016first}, there is a high chance that it will be
introduced to the remaining countries in the near future. Such invasions
can have devastating effect on the economy and livelihood of farmers.
Moreover, invasion in Mainland Southeast Asia in particular is a serious
threat to China, the largest producer of tomato, and Australasian
neighbors.

%% Furthermore, no earlier work has studied the entire
%% Southeast Asia region, particularly in the context of agricultural crops.

\paragraph{Summary of results.} 
%% We studied \tuta{} invasion of Bangladesh using the proposed model. 
Our analysis with respect to historical invasion records indicate two
possibilities of spread of \tuta{} in Bangladesh, one with trade as the
dominant pathway and the other without. In particular, even with slow
self-mediated dispersal, the pest could have rapidly expanded its range
aided by domestic tomato trade. For the rest of the study region our models
predict a faster southward spread to Thailand, Malaysia and Singapore than
eastward spread in Mainland Southeast Asia due to higher trade activity in
the former region. Country specific analysis shows that once introduced to
a major production area, the pest will spread all over the country within
two to three years. However, lack of information on trade interactions
between countries makes it difficult to predict pest movement between
countries. We also demonstrated the efficacy of intervening at the trade
level; quarantining few key production areas can contain the spread or
significantly reduce the spread rate.

%%
\section{Methods}
%%
\paragraph{Data.} The global datasets used in the model and for analysis are
described in Table~\ref{S:tab:data} of the supplement. Country specific data on
seasonal production, consumption, processing and trade was
obtained from websites of agriculture ministries, research articles and technical
reports (Table~\ref{S:tab:countryData} in SI). Almost
all the datasets used are openly available. Some information was
provided by local contacts in Bangladesh (for e.g., \tuta{} incidence
reports in Table~\ref{S:tab:bgdData}), Vietnam and Cambodia.
%%
\paragraph{\tuta{} biology.}
The tomato leafminer exhibits a short life cycle of about 24--38 days
(temperature at $25\pm3^\circ$C), from egg to adult, as it is a
multivoltine species with overlapping generations in the
field~\cite{guedes2012tomato}. This species causes serious damage to
numerous solanaceae crops such as eggplants, potatoes, and especially
tomato crops~\cite{sylla2018}. It penetrates into tomato leaves, stems, or
fruits, wherein it feeds and develops by creating conspicuous mines as well
as galleries. \tuta{} additionally restricts tomato plant growth by feeding
on the growing tips. Considering the warm weather throughout the year,
particularly in the dry season, the study region presents ideal conditions
for rapid development and spread of \tuta{}. Pest risk
analysis~\cite{tonnang2015identification} shows that the Ecoclimatic Index
for this region is above~$50$ (highly suitable). Spatial distribution
assessment survey of \tuta{} eggs has shown its high dispersive capacity in
tomato producing areas~\cite{martins2018assessing}. The dispersion in a
tomato cultivation starts mainly at the periphery and
the pest is able to migrate between tomato farms to generate egg aggregation at the
crop edges. The pest spread behavior among seasonal crop resources
is often non-random and directional~\cite{martins2018assessing}.
Sylla~et~al.~\cite{sylla2018} analyzed host preference of~\tuta{} in France
and Senegal. While the highest preference is for tomato, it can survive
well on eggplant and potato, which happen to be major vegetable crops in
the study region. However, since \tuta{} primarily attacks leaves of
eggplant and potato, the chance of the pest spreading through trade of
these crops seems to be low.
%%
%% Each cell~$v$ has the following time varying attributes: \aacomment{these
%% are not necessary: NDVI~$\veg(v,t)$,
%% temperature~$\temp(v,t)$,} monthly production of preferred host crops and
%% consumption~$\consume(v,t)$, where~$t$ corresponds to a month.
%% \aacomment{This should get into Tonnang discussion: Since NDVI
%% data is available at a much finer resolution (see Table~\ref{tab:data}),
%% for each cell, we assign the maximum of the NDVI data points that fall in
%% this cell for that month.} For production, we consider the following
%% vegetables: tomato, potato, eggplant in metric tonnes (see
%% Section~\ref{sec:biology} on choice of host crops). Annual country-level
%% production data (in volume) is spatially disaggregated using SPAM data
%% on spatial distribution of vegetable production. \aacomment{needs
%% explanation. statistical analysis should come here}. Further, for each cell,
%% this production was distributed temporally-- one value for each month of
%% the year. We observed that precipitation is a primary driver of production
%% in the study area; during wet months, the amount of production is
%% considerably less. \aacomment{statistical analysis should come here} (See Supplementary Information).
%% \aacomment{pending: consumption, import/export}

%%
\paragraph{Multi-pathway spread model.} We developed a stochastic
multi-scale propagation model to simulate the multi-pathway spread
of~\tuta{}. Key concepts are illustrated in Figure~\ref{fig:concept}. The
study region is divided into cells by overlaying a grid
(0.25\textdegree~$\times$~0.25\textdegree). Each cell is in one of the
three states: susceptible ($S$) denoting pest free state, exposed ($E$)
denoting that the pest has been introduced but the population has not yet
built up to influence other cells, and infectious ($I$) denoting that the
pest has established and the cell can influence its neighbors. The cell
states are updated in discrete time steps, where each step~$t$ corresponds
to a month.  The probability that a cell~$v$ transitions from state~$S$
to~$E$ is determined by (i)~suitability of the cell for \tuta{} to
establish at that time step~$\suitable(v,t)$ and (ii)~influence of
``neighboring'' cells in state~$I$ depending on the pathway. An exposed
cell transitions to state~$I$ after a latency period of~$\ell$ time steps.
This is the time required for the population to build up to infect other
cells.  Once the pest has established in a cell, the cell remains infected
forever, a fair assumption considering that, historically, eradication of
\tuta{} has not been successful\footnote{The only exception is United
Kingdom where the pest was detected early and
eradicated~(\url{https://gd.eppo.int/reporting/article-340}).}.  The
infectiousness of a cell~$\infest(v,t)$ is modeled as a linear function of
host presence at time~$t$, for which we use the weighted sum of production
volume of tomato, eggplant, and potato in that cell at time~$t$. The
weights correspond to relative carrying capacity of \tuta{} on the three
hosts~\cite{sylla2018}.

%%
\begin{figure}[t]
\centering
\begin{subfigure}[b]{.4\textwidth}
    \includegraphics[width=1.1\textwidth]{figs/model_schematic.pdf}
\caption{\label{fig:concept}}
\end{subfigure}
%%
\begin{subfigure}[b]{.56\textwidth}
    \includegraphics[width=1.05\textwidth]{figs/pipeline.pdf}
\caption{\label{fig:pipeline}}
\end{subfigure}
\caption{\textbf{The multi-pathway model.} (a)~The network structure,
pathways and dynamics are captured in the illustration.
%% , which is
%% approximately~$27.8\mathrm{km}\times\mathrm{27.8km}$ at the equator. 
%% These
%% dimensions are comparable to that used in the cellular automata model of
%% Guimapi~et.~al.~\cite{guimapi2016modeling} ($25\mathrm{km}\times25\mathrm{km}$). Details of locality construction are in Section~\ref{S:sec:locality}. 
%% Long distance human-mediated dispersal is modeled
%%las the spread between localities through. 
Also shown are the states and factors
that influence state transitions: infectiousness of a neighbor, suitability
of the cell for pest establishment, pathway parameters and latency period.
\label{fig:modelConcept}
(b)~\textbf{Pipeline.} The process of constructing the spatio-temporal
network of cells is outlined. Key modules are highlighted along with 
input data.}
\end{figure}
%%

There are three pathways by which a cell can become infected:
short-distance dispersal, local human-mediated dispersal and long-distance
dispersal. Short-distance dispersal captures the spread through natural
means; from an infested cell to cells in its Moore neighborhood of
range~$\mooreRange$.  The probability that a susceptible cell gets exposed
(E) at time step~$t$ through short-distance spread is as follows:
%%
\begin{linenomath}
\begin{align}\label{eqn:pshort}
    \pshort(v,t)=\suitable(v,t)\bigg(1-
    \exp\Big(-\asd\sum_{v'\in\moore_v(\mooreRange)}\infest(v',t)\Big)\bigg),
\end{align}
\end{linenomath}
%%
The probability depends on the suitability of the cell~$\suitable(v,t)$,
infestation level of each neighboring cell in the Moore neighborhood with
range~$\mooreRange$,~$\infest(v',t)$ and the scaling factor,~$\asd$, which
is the transmission rate for this pathway. The function form is explained
in Section~\ref{S:trans}.

For human-assisted spread we identified large urban areas in the region
which we refer to as {\it localities} (Figure~\ref{fig:modelConcept}) and
considered interactions within and between localities. These areas act as
attractors of vegetable flows due to high consumption or production and
house the necessary infrastructure: wholesale markets, traders and
distributors.  Each \emph{locality} consists of all grid cells which are
within a certain distance (determined by \emph{locality radius}) from its
corresponding center. Local human-mediated dispersal is modeled as the
spread between cells belonging to a locality.  Every cell~$v$ is influenced
by cells in its locality~$\locality$ based on their infectiousness.  The
expression is similar to that in~\eqref{eqn:pshort}, but with cells in the
locality instead of the Moore neighborhood.
%%
\begin{linenomath}
\begin{align}\label{eqn:plocal}
    \plocal(v,t)=\suitable(v,t)\bigg(1-
    \exp\Big(-\afm\sum_{v'\in\locality}\infest(v',t)\Big)\bigg),
\end{align}
\end{linenomath}
%%
where~$\afm$ is the scaling factor. The details of
locality construction are provided in Section~\ref{S:sec:locality} of SI.

Long-distance human-mediated dispersal corresponds to spread through trade
between localities. For this purpose, we considered only tomato trade as
there is not much evidence of \tuta{} spreading through trade of other
hosts. We modeled domestic trade using a gravity model approach accounting
for tomato production, processing, imports and exports in each locality and
the travel time between localities. 
The probability of spread is directly proportional to the trade
flow~$F_{ij}$ from one locality ($i$) to another ($j$).
Suppose cell~$v$ belongs to locality~$i$. Then, the probability of cell~$v$
transitioning from~$S$ to~$E$ due to long-distance dispersal is given by:
%%
\begin{linenomath}
\begin{align}\label{eqn:plocal}
    \pld(v,t)=\suitable(v,t)\bigg(1-
    \exp\Big(-\ald\sum_{j\ne i}\sum_{v'\in\locality(j)}F_{ji}\infest(v',t)\Big)\bigg),
\end{align}
\end{linenomath}
%%
where~$\ald$ is the scaling factor.
The model parameters and their values are summarized in
Table~\ref{tab:param}. 
%%
\begin{table}[t]
\caption{Model parameters and their values.\label{tab:param}}
    \centering
	\small
\rowcolors{2}{white}{gray!15} % For this to work, put \PassOptionsToPackage{table}{xcolor} before \documentclass
    \begin{tabular}{p{.093\textwidth}p{.33\textwidth}p{.5\textwidth}}
    %{cp{.25\textwidth}p{.25\textwidth}lr}
		\hline		
		Parameter & Description & Value/range \\
\hline		
\hline
$\mooreRange$ & Range of Moore neighborhood & $\{1,2,3\}$ corresponding to
spread per month of 
approximately~$25$kms,~$50$kms and~$75$kms
respectively~\cite{guimapi2016modeling,martins2018assessing}. \\
%% $\suitable$ & Suitability threshold \\
$\ell$ & Latency period to transition from $E$ to $I$ & $\{1,2,3\}$ months
based on time for the pest to complete life cycle (\tuta{} biology in
Methods). \\
Monthly production & Disaggregation of annual production to monthly values
& \emph{Uniform} throughout the year or \emph{seasonal} based on regression
analysis (Methods). \\
$\beta$ & Gravity model distance function exponent & $\{0,1,2\}$ \\
$\kappa$ & Gravity model distance function cut-off & Between $4$ to $16$ hours
of travel time. \\
Seed & Location and time of initial infestation & Scenarios based on
countries (Table~\ref{S:tab:seed})\\
Locality radius & Determines cells assigned to a locality & 100kms \\\hline
$t_s$ & Time step of initial infestation & $\{3,4,5\}$ corresponding
to March, April and May respectively based on first report in
Bangladesh~\cite{hossain2016first}. \\
%% $\infest$ & Infectivity of a cell based on amount of production & \\
$\asd$ & Short-distance spread scaling factor & In the interval $[0,500]$.\\
$\afm$ & Local human-mediated dispersal scaling factor & In the interval $[0,500]$.\\
$\ald$ & Long distance spread scaling factor & In the interval $[0,500]$.\\
\hline
\end{tabular}
\end{table}
%%
\paragraph{Network construction.} 
Figure~\ref{fig:pipeline} provides a schematic of the network construction.
The first step was to estimate monthly production volume of tomato,
eggplant and potato for each cell. We estimated annual production in each
cell followed by disaggregation to monthly production. The annual
production was estimated using the vegetable production available at the
highest resolution for each country (at the level of province to just one
value for the entire country) and a synthetic dataset called Spatial
Production Allocation Model (SPAM)~\cite{spam}. For monthly production, we
used linear regression to model the production rate as a function of
precipitation, temperature, and elevation. Seasonal tomato and eggplant
production data for different regions of Philippines was used. For most of
the other countries only qualitative information on seasonal of production
is available (Table~\ref{S:tab:countryData}). The regression function was
applied to locations of these countries where this information is
available and visually compared available data. The details are in
Section~\ref{S:sec:prod} of the supplement.

To model locality-to-locality trade of host crops, we applied the approach
of
Venkatramanan~et~al.~\cite{venkatramanan2017towards,venkatramanan2019modeling}
with some modifications. While there is strong evidence of tomato trade as
a possible pathway of \tuta{} spread~\cite{biondi2017}, there is no
evidence of a similar role of eggplant or potato trade.  We modeled the
flow of fresh tomato crop between markets based on the following
assumptions: (i)~The total outflow from a city depends on the amount of
produce in its surrounding regions and imports from countries outside the
focus region at time~$t$, and (ii)~the total inflow depends on total
consumption, processing demand, and exports from the city to countries
outside the focus region. The details are in Section~\ref{S:sec:netcon}.
Since no data is available on volume of trade between markets, to validate
the modeled trade networks, we gathered qualitative information from
reports and research articles which show evidence of tomato or vegetable
flow between market pairs. The results of this analysis is in
Section~\ref{S:sec:tradeFlows}.  Trade between countries of the focus
region was not modeled as there is no adequate information on ports of
entry or monthly flow volumes.

\paragraph{Parameterization and experiment design.}
The goodness of fit of a parameter instance was determined by comparing the
simulation output with \tuta{} incidence reports. We have incidence
information from eight locations in Bangladesh where pheromone traps were
installed (Figure~\ref{fig:bgdClassA} and Table~\ref{S:tab:bgdData}). The
spread was simulated with infestation starting from the location of first
report (Panchagarh, Bangladesh). For each cell, empirical probability that
it is in state~$I$ at time~$t$ was computed (averaged over 100
repetitions). The output was compared with ground truth using a similarity
function adapted from~\cite{carrasco2010unveiling}.  Let~$v$ be a reporting
cell and~$t_v$ denote the month of actual report of pest presence.  To
account for uncertainty in reporting, we consider a time
window~$U_\tau=[t_v-\tau,t_v+\tau]$ during comparison, where~$\tau$ is the
uncertainty parameter. We set~$\tau=2$, i.e., error within $\pm2$ months is
tolerated.  Supposing~$\reportingCells$ is the set of cells corresponding
to ground truth,  and ~$p(C,v,t)$ is the empirical probability that cell~$v$
is infected at time~$t$ for the input configuration~$C$, then, the similarity~$\similarity$ is
given by,
%%
\begin{linenomath}
\begin{align}\label{eqn:similarity}
    \similarity(C)=\sum_{v\in\reportingCells} \Big(\sum_{t\in U_\tau}p(C,v,t)
    + \sum_{t\notin U_\tau}\big(1-p(C,v,t)\big) \Big)\,.
\end{align}
\end{linenomath}
%%
Therefore, the maximum possible value
of~$\similarity$ is~$8$ (exact match with ground truth) and minimum is~$0$.

The range or values of model parameters are in Table~\ref{tab:param}.
Parameter space exploration was conducted in multiple iterations.  First,
we coarsely sampled the space. With model parameters as independent
variables and the goodness of fit measure defined in~\eqref{eqn:similarity}
as the dependent variable, we used Classification and Regression Trees
(CART) approach to identify subspaces for which similarity score was high
and rejected subspaces for which similarity was low. Based on this, in the
subsequent phases, we refined our search to improve the parameterization.
More details of the CART analysis is presented in Section~\ref{S:sec:cart}.
Simulations were performed on more than~$500,000$ parameter combinations
using a high performance computing cluster. Configurations with high
similarity score $\similarity(C)\ge6$ ($\ge75\%$ of maximum value) were
chosen for further analysis.

%% In the second phase,
%% we applied the select models from the previous phase to the rest of the
%% region to study how various conditions affect the nature and rate of
%% spread: different pest introduction scenarios, seasonality of production
%% and trade, and interventions or the lack thereof.
%% 
%% re

\paragraph{Analysis of spread pattern.} The objective here is to analyze
the variability in the simulation outcomes within the set of best fit
configurations. We leverage well-known machine learning techniques in a
novel way to address this question. The methodology is 
outlined in Figure~\ref{fig:clusterOutline}. First, we clustered
the simulation outputs (time and cell indexed empirical probabilities) of
selected configurations from the parameterization phase. This step captures
the variability in outcomes; simulation outputs belonging to different
clusters can be considered to be significantly different from one another.
In the second step, we attempt to unearth relationships between model
parameters and cluster membership. Our approach was to cast this as a
classification problem using CART with model parameters as the independent
variables and cluster index as the dependent variable. The relationsips
were inferred from the decision tree that results from the algorithm.
To avoid any bias introduced by the clustering algorithm, we applied more
than one method -- hierarchical (agglomerative SPSS) clustering and $k$-means
algorithm (Pyclustering). In both cases, we used the Euclidean distance as the distance
measure to compare two simulation outputs. The analysis was repeated for
different values of~$k$, the number of clusters. More details are provided
in Section~\ref{S:sec:cluster} of the supplement.
%%
\begin{figure}[htb]
    \centering
    \includegraphics[width=.8\textwidth]{figs/spread_analysis.pdf}
    \caption{Outline of the process of analyzing the multi-pathway spread. \label{fig:clusterOutline}}
\end{figure}
%%
\section{Results}
\paragraph{Variability in spread pattern.}
The clustering of the configurations selected during the parameterization
phase (approximately 8000 of them) reveals two distinct spread patterns
primarily determined by the pathway parameters.  The first class of models
(Figure~\ref{fig:bgdClassA}), referred to as Class~A, are characterized by
the absence of long-distance human-mediated spread ($\ald$ negligible) and
brisk spread between geographically adjacent cells, driven by Moore
range~$\mooreRange$, latency period~$\ell$ and short-distance scaling
factor~$\asd$. In contrast, in Class~B models
(Figure~\ref{fig:bgdClassB1}), the long-distance pathway~($\ald$) plays a
significant role  and there is relatively slow spread between
geographically adjacent neighbors. Both hierarchical clustering and
$k$-means clustering (Figures~\ref{S:fig:cartAgglomerative}(b)
and~\ref{S:fig:cartkmeans}(a)) are consistent in this regard.

%%
\begin{figure}[t]
    \centering
\begin{subfigure}[b]{.3\textwidth}
    \includegraphics[width=\textwidth]{../cellular_automata/results/contour/BGD_model-A.pdf}
    \caption{Class~A ($\ald=0$) \label{fig:bgdClassA}}
\end{subfigure}\hspace{.5cm}
\begin{subfigure}[b]{.3\textwidth}
    \includegraphics[width=\textwidth]{../cellular_automata/results/contour/BGD_model-B_m1_l3.pdf}
    \caption{Class~B ($\ald>0$) \label{fig:bgdClassB1}}
\end{subfigure}
\begin{subfigure}[b]{.32\textwidth}
    \centering
    \includegraphics[width=0.9\textwidth]{../cellular_automata/results/rf/rf_importance_all_mdi.pdf}
    \includegraphics[width=0.9\textwidth]{../clustering/results/agglomerative/rf_k_agglomerative_mse.pdf}
    \caption{Parameter importance \label{fig:rfA}}
\end{subfigure}
\caption{\textbf{Spread in Bangladesh.}The contour plots show the simulated
spread starting from the location of first report in Panchagarh district
for 12 months. For the purpose of plotting, the time of infection for a
cell is the minimum time step~$t$ such that the empirical probability that
the cell is infected by time~$t$ is $\ge0.8$. Also highlighted are the
eight monitored locations and the localities applied in the model. The
colors of the monitored locations correspond to the month of report
relative to the first report (Panchagarh). Two distinct spread patterns
emerged from the cluster analysis. (a) and (b) show representative spreads
observed for each class. The similarity ($\similarity$) in each case was
$>6.5$.}
%f (a)~\textbf{Class~A.} Spread pattern long-distance component playing a
%% negligible role. We
%% observe a steady radial spread.  (b)~\textbf{Class~B.} Spread pattern with
%% long-distance jumps. Here, the radial spread is much slower.}
\end{figure}
%%

Class~A spread pattern does not capture the gap between the time of first
report (Panchagarh) and the report in Gaibandha district
(Figure~\ref{fig:bgdClassA}). Even though the distance between the two
locations is only $185$kms, the latter reported the presence only after~10
months of first report suggesting that self-mediated spread might have been
much slower. In the model output on the other hand, the corresponding cell
gets infected between the second and fourth months.  In Class~B, this
location is infected much later in comparison. However, the eastward spread
towards the location Jaintiapur is slower than what was observed
(Figure~\ref{fig:bgdClassB1}). Even though Panchagarh is quite far from
this location, pest presence was reported by February 2017, just nine
months from the first report.  
As a baseline, we also simulated the spread using the cellular automata
model developed by Guimapi~\cite{guimapi2016modeling} for Bangladesh. The
spread pattern is similar to Class~A as the model does not account for
long-distance hops. However, the predicted rate of range expansion is much
higher than our models (see Section~\ref{S:sec:guimapi} for model details
and results).

\paragraph{Importance of model parameters.} 
how parameters affect similarity score?

how parameters affect spread pattern?

cluster size

We studied sensitivity of the
spread pattern to model parameters by applying Random Forest learning
method. Since the clustering captures the variations in spread patterns,
the algorithm was applied with model parameters as independent variables
and cluster index as the dependent variable. The results are in
Figure~\ref{fig:sensitivity}.

The long distance pathway parameter~$\ald$ is by far the most important
parameter (Figure~\ref{fig:rfAll}); both spread rate and pattern critically
depend on it. Moore range ($\mooreRange$) and latency period ($\ell$) are
important as they affect the rate of radial spread. The start month ($t_s$)
is also important due to two reasons. Firstly, the distance between two
time shifted simulation outputs can be large. Secondly, outputs are
sensitive to seasonal variations or temporality of the network. Focussing
on Class~A models, the main drivers of the spread pattern are latency
period and Moore range (Figure~\ref{fig:rfA}). Since the simulation
duration represents more than one year of spread, seasonality (season) does
not have much effect on the spread. The distance parameters~$\beta$
and~$\kappa$ do not influece the spread as~$\ald=0$. Seasonal production on
the other hand is very important in Class~B models as it influences trade
flows.

\paragraph{Scenarios of pest introduction to countries in Southeast Asia.}
To identify routes of introduction to other countries in the region, we
applied both Class~A and Class~B models. The starting point of the spread
corresponds to Panchagarh district (Figure~\ref{fig:bgdClassA}).  We also
analyzed the international tomato trade network (Section~\ref{S:sec:tomnet}
of supplement) to assess the risk due to imports from \tuta{} infested
countries outside this region. Both model classes strongly indicate that
\tuta{} is already present in parts of Myanmar (curves corresponding to
timestep 24 or two years from first report). Also, the pest is likely to
enter Thailand from Myanmar, and subsequently move to Laos and Vietnam as
it spreads eastwards and to China when it spreads northwards. From
Thailand, spreading southward, it will enter Malaysia and subsequently
enter Indonesia (Figure~\ref{fig:msaClassAB}).

%%
\begin{figure}[ht]
    \centering
\begin{subfigure}[b]{.47\textwidth}
    \includegraphics[width=\textwidth]{../cellular_automata/results/contour/MSA_model-A_m2_l1.pdf}
    \caption{Class~A\label{fig:msaClassA}}
\end{subfigure}\hspace{.5cm}
\begin{subfigure}[b]{.47\textwidth}
    \includegraphics[width=\textwidth]{../cellular_automata/results/contour/MSA_model-B_m1_l3.pdf}
    \caption{Class~B\label{fig:msaClassB}}
\end{subfigure}
\caption{\textbf{Spread pattern in Mainland Southeast Asia without
accounting for trade between countries.} The contour plots show the
simulated spread starting from northern Myanmar for 120 time steps or 10
years. Representative simulation outputs for Class~A and Class~B models are
shown. With {Class~A}, spread is faster eastward than southward and the
other way round in the case of Class~B.\label{fig:msaClassAB} }
\end{figure}
%% %%   

From trade perspective, Malaysia and Singapore are important hubs with
tomato imports from \tuta{} infested regions. There is a possibility that
\tuta{} is directly introduced to these regions.  However, in both cases,
the import volume is very low. Also, the introduction risk depends on the
preventive measure taken by the exporting countries. With respect to both
trade and natural pathway, there is a low chance that the pest will be
introduced into Philippines from neighboring countries.  This is because,
it does not share its borders with any country in the region and also,
there is no evidence of tomato trade with rest of the countries. However,
human mobility is a possible pathway. The Middle East is the top
destination for Filipino workers~\cite{rodriguez2011philippine}.
Therefore, there is a possibility of introduction through human mobility.
%%
\paragraph{Predicted spread is model and region dependent.} In the case of Class~A
models, the eastward spread is faster than southward spread
(Figure~\ref{fig:msaClassA}). This is mainly because the Moore neighborhood
is smaller at the narrow region in the south of Myanmar and Thailand
bordering Malaysia.  However, in the case of Class~B
(Figure~\ref{fig:msaClassB}), the spread is much faster in the same region
aided by domestic trade flows from northern and central Thailand to the
southern region.  Class~A spread pattern predicts that within the next 4-5
years much of the northern part of Mainland Southeast Asia will be invaded.
Class~B spread pattern predicts that in the same period \tuta{} will spread
all over Malaysia and Singapore.  However, the rate of spread observed is
slower than that observed in Bangladesh for both classes. Also, even though
the models exhibited similar rate of spread for Bangladesh, we observed
high variance in intensity of infestation as well range expansion for the
rest of the region. The results are in Figure~\ref{S:fig:spreadRate}. The
reason for slow spread is as follows.  Bangladesh has the highest tomato
volume per country surface area ($\approx2.5$tonnes/km$^2$).  The next
country is Vietnam ($\approx1.5$tonnes/km$^2$). Therefore, in the case of
Bangladesh, not only is the extent of infestation in a
cell~$\infest(\cdot)$ typically high, but also, since it is a densely
populated country, most cells have vegetable production. Hence, the rate of
spread is much higher for relatively lower values of pathway parameters and
Moore range. Also, we observed a strong dependence on Moore range
(Figure~\ref{S:fig:spreadRateB}).  In countries with larger area, the
production is scattered. Therefore, lower the Moore range, the slower the
spread.

\paragraph{Seasonal production and consumption drive commodity flow.}
Henceforth, we focus on long-distance dispersal, and therefore, restrict our
discussion to Class~B models. Analyzing qualitative data from multiple sources
, we inferred that the
cropping pattern in this region depends primarily on two factors:
seasons--dry and wet, and elevation--highland (upland) and lowland. Also,
barring few exceptions, the production of the considered host crops peaks
in the winter and is lowest during the rainy season (monsoon).  To predict
seasonal production, we used linear regression to model the production rate
as a function of precipitation, temperature, and elevation.   The regression
results showed that precipitation was a statistically significant predictor
($p<0.001$).  Regional tomato and eggplant
production data is available for Philippines.

few major production centers

major consumption centers get vegetables from long distances as they cannot
satisfy the demand

precipitation as a seasonal production predictor.

district level is required

few production areas and many consumption areas showing hub and spoke.


\paragraph{Influence of domestic trade on spread pattern and rate.}
For the country-specific studies, the starting location was decided based
on our analysis of possible entry points through different pathways
(Section~\ref{S:sec:seeds} in the supplement).  We observed the following
common spread pattern.  When the invasive species is introduced to a
country, dispersal is slow until the invasion front reaches a {production
source}. Once it establishes at a source, the spread is very fast.
Depending on the country, within~12 to~24 timesteps (or an year or two), it spreads to almost all major localities of the country
(see Figure~\ref{fig:thlBContourInt} for example). Production areas which
are very close to high-consumption localities (large urban areas) are
particularly vulnerable. Since local production typically does not satisfy
demands of such localities, they have high inflows from other production
areas and possibly from other countries. As a result, these localities are
the quickly infected. Once introduced to such localities, farmer--market
interactions (local human-mediated dispersal) can facilitate the
introduction of the pest to nearby production regions where it can
establish and proliferate.

%%
Given that monitoring and quarantining are both resource intensive and
potentially disruptive, developing strategies that involve few locations,
yet provide near-optimal control is a goal for modelers.  Market-level
phytosanitary measures in terms of import restrictions have been undertaken
by countries~\cite{USDA2012}. Here, we evaluated a simple strategy of
containing the spread through the trade pathway. Localities associated with
high annual outflows were identified (at most four in each country). As
discussed earlier, pest establishment in these areas can potentially lead
to rapid range expansion. The outflow from the targeted localities was cut
off to mimic control at the trade/market level. In the strictest sense,
this can be implemented by restricting trade of host crops. But, it is
possible that phytosanitary measures have the same effect.
Figure~\ref{fig:spread} shows results for two countries.  More results are
present in Figure~\ref{S:fig:intervene} in the supplement.  Consistently,
across countries, we observed a significant reduction in range expansion as
well as intensity of spread.  Besides, as seen in
Figure~\ref{fig:thlBContourInt}, stifling these flows localizes the spread
that resembles those of Class~A models, but with much less intensity.
%%
%% Given the importance of the long-distance human-mediated pathway, we
%% studied the effect of controlling this pathway in mitigating the spread.
%% Monitoring localities by setting up pheromone traps in production areas and
%% markets, and quarantining affected areas is one way to accomplish this.
%%
\begin{figure}[ht]
\begin{subfigure}[b]{.28\textwidth}
\includegraphics[width=\textwidth]{../cellular_automata/results/contour/TH_model-B_precip1_m1_l3.pdf}
\caption{Thailand without intervention\label{fig:thlBContour}}
\end{subfigure}
\begin{subfigure}[b]{.28\textwidth}
\includegraphics[width=\textwidth]{../cellular_automata/results/contour/TH_model-B_precip1-out-100_m1_l3.pdf}
\caption{Thailand with intervention\label{fig:thlBContourInt}}
\end{subfigure}
\begin{subfigure}[b]{.43\textwidth}
\includegraphics[width=\textwidth]{../cellular_automata/results/dist_inf_plots/TH_dist_prob_B_box.pdf}
\caption{Thailand: range of expansion (all Class B models)\label{fig:thlBContourBox}}
\end{subfigure}
\begin{subfigure}[b]{.28\textwidth}
\includegraphics[width=\textwidth]{../cellular_automata/results/contour/VN_model-B_precip1_m1_l3.pdf}
\caption{Vietnam: without intervention\label{fig:vnmBContour}}
\end{subfigure}
\begin{subfigure}[b]{.28\textwidth}
\includegraphics[width=\textwidth]{../cellular_automata/results/contour/VN_model-B_precip1-out-100_m1_l3.pdf}
\caption{Vietnam: with intervention\label{fig:vnmBContourInt}}
\end{subfigure}
\begin{subfigure}[b]{.43\textwidth}
\includegraphics[width=\textwidth]{../cellular_automata/results/dist_inf_plots/VN_dist_prob_B_box.pdf}
\caption{Vietnam: range of expansion (all Class B models)\label{fig:vnmBContourBox}}
\end{subfigure}
\caption{\textbf{Rate and pattern of spread with and without intervention.}
Representative spread dynamics of Class~B models ($\mooreRange=1, \ell=3$)
for two countries. More plots are in Figure~\ref{S:fig:intervene}. In each
case, a cell close to a high production region was seeded. The first column
corresponds to spread for 48 months after introduction. The colors indicate
the time interval at which there is at least a 50\% chance that a location
will be infected.The second column corresponds to spread after cutting off
flows from chosen localities. The third column shows average spread with
respect to origin of infection for all Class~B models. The cells are binned
based on their distance from the origin of infection. Given time step~$t$
(48), let $\Pr(v,\le t)$ be the probability that cell~$v$ is in state~$I$
by time~$t$. For each configuration, we computed the ``total
infection'' for every bin at time~$t$ by aggregating~$\Pr(v,\le t)$ for
each~$v$ in the bin. The red points referred to as ``max'' correspond to
the total number of cells in each bin, which is also the maximum possible
accumulated probability for that bin.
\label{fig:spread}}
\end{figure}
%%     \textbf{Interventions.} The effect of
%% reducing outflows from major production regions is shown for Class~B
%% models. These are representative results for Philippines. Plots for other
%% countries are in Figure~\ref{S:fig:intervene}. In the simulations, a high
%% production region (Northern Mindanao) was seeded. (a)~\textbf{Without
%% intervention.} Simulations indicate that for the given initial conditions,
%% there is a high chance that all major production areas of Philippines will
%% be affected within 24 months.  The colors indicate the time interval at
%% which there is at least a 50\% chance that a location will be infected.
%% (b)~\textbf{With market level intervention.} The spread without the
%% influence of long-distance flows from localities corresponding to high
%% production areas (Northern Mindanao and Central Luzon). We observe a delay
%% of more than two years in the introduction to the northern part due to this
%% intervention.  (c)~\textbf{Spread with respect to origin of infection.}
%% We compare the effect of reducing the long-distance flow the
%% above-mentioned regions by 50\% and 100\%. The results shown correspond to
%% all Class~B instances with Moore range~$\mooreRange=1$. \label{fig:intervene}}
%% Under the assumption that appropriate control
%% measures are taken, we reduce the incoming and outgoing edge weights
%% by~$50\%$ of their original value. The resulting model was compared with
%% the case where no action is taken (Section~\ref{sec:predict}).
%% 
%% Farm-level \aacomment{not sure about this}
%% \begin{itemize}
%%     \item suitability threshold is varied
%%     \item Scenario 1 where for all cells it is done
%%     \item Scenario 2 where for it is done in key tomato growing areas
%%     \item Scenario 3 where it is done only near localities
%% \end{itemize}
%% 
%% Market-level
%% \begin{itemize}
%%     \item Scenario 1: we identify localities which are hubs (lots of
%%     outflow) and reduce their flow.
%% \end{itemize}
%% %%
%% \begin{figure}[ht]
%%     \centering
%%     \begin{subfigure}[b]{.47\textwidth}
%%         \includegraphics[width=\textwidth]{figs/spread_farm_level_interventions.png}
%%     \caption{\label{fig:spreadFarmLevel}}
%%     \end{subfigure}
%%     \begin{subfigure}[b]{.47\textwidth}
%%     \includegraphics[width=\textwidth]{figs/spread_market_level_interventions.png}
%%     \caption{\label{fig:spreadMarketLevel}}
%%     \end{subfigure}
%%     \caption{Effect of interventions}
%% \end{figure}

%%
\section{Discussion}
The variability in the spread patterns that explain the incidence reports
exposes the lack of understanding of the pathways of spread.
Nevertheless, the analysis does strongly indicate the role of
human-assisted spread of \tuta{}. The pest was reported in May~2016 in the
nortwestern part of Bangladesh bordering India. The region is among the top
three tomato producers in the country. By the beginning of the next
production season, \tuta{} was found in almost every major urban region.
Similar correlation between tomato trade and \tuta{} spread was observed in
Nepal~\cite{venkatramanan2019modeling}. Studies on
self-mediated spread (flying capability or by wind) can definitely help
estimate more accurately the rate of self-mediated spread. It is also
important to consider alternate scenarios of introduction. We recall that
the far eastern part of Bangladesh (locality Jaintiapur in
Figure~\ref{fig:bgdClassB1}) reported pest presence nine months after the
first report. In a typical Class~B spread pattern, however, this region was
not infected within that time frame mainly because of week trade flows to
that region. However, in January 2017, \tuta{} was reported from
Umiam~\cite{sankarganesh2017}, Meghalaya in India, about 100kms from
Jaintiapur. It also happens to be near an important trade route from India
to Northeastern Bangladesh. Therefore, it is possible that multiple
incursions took place. It is possible that analysis of other countries or
regions can reduce this variability. But, caution needs to be exercised
before applying analysis of one region to another as trade dynamics,
production methods and as a result pathways can wary widely from one region
to another.

We recall the discussion on slow predicted rate of spread in Mainland
Southeast Asia compared to the observed rate in Bangladesh. One reason for
this could be the unaccounted trade flows between countries. International
trade within this region is not documented well. Historically,
international trade has played a strong role in the spread of \tuta{}
between countries. For example, the pest was first reported by India
in~2014~\cite{sridhar2014new,kalleshwaraswamy2015occurrence}. By
early~2016, it was discovered in the Kathmandu area of
Nepal~\cite{bajracharya2016first}, the northern part of Bangladesh in
May~2016~\cite{hossain2016first}. Both countries import significant volume
of tomato from India. However, there has been no report from Pakistan,
another neighbor which does not import tomato from India. It is possible
that the pest is present and undetected in this country. But, it is clear
that Pakistan is aware of the threat\footnote{The looming threat of the
deadly tomato leafminer~(\url{https://www.dawn.com/news/1420206}).}
indicating that even if it is present, it is not widespread. There are
similar examples outside the region such as its slow advance from South
America to Central America, or the fact that it is not reported in China
despite being present in neighboring Central Asian countries since~2015.
Hence, it is critical to address the data gaps concerning international
trade.

While, several IPM strategies have been suggested for managing \tuta{},
hardly any work has been done in designing effective interventions at the
trade level. Designing phytosanitary measures targetted towards market and
vehicles of transportation for preventing introductions (or
reintroductions) is therefore a promising research direction. Some
countries have already taken measures in this regard. The Animal and Plant
Health Inspection Service of the United States Department of Agriculture
(USDA-APHIS) has instituted quarantine regulations for imports from regions
where the pest is present~\cite{USDA2012}. Identifying the optimal set of
nodes in a network to reduce infectious disease spread is a widely studied
topic~\cite{madar2004immunization}. There are very few works that apply
such techniques to invasive species spread~(Nopsa et
al.~\cite{nopsa2015ecological} for example). As the world moves towards
concentrated and specialized agricultural production, focusing on this
aspect becomes increasingly important.

need to say take away points
\begin{enumerate}
    \item modularity of the parameterization and analysis part needs to be
    stressed
    \item the clustering paper citation \cite{cassandras2000clustering}
    \item need sample flows for calibration. need to study heterogeneity
    \item need function for relation between flow and pest spread
    \item what is enough for modeling trade? at minimum what is
    required?
    \item SIS Epidemics in Multilayer-based Temporal Networks
\end{enumerate}

\paragraph{Impact.} Traditionally, crops such as tomato have been grown in
the dry season, which is usually during winter in most parts of Southeast
Asia. However, over the past decade, due to rising demand and opportunities
to export, there has been a thrust towards year-round production using
protected cultivation methods and resilient varieties~\cite{ali2001}.
Tomato production and internal trade has steadily increased in this region
(See Figure~\ref{S:fig:trends} in supplement). In comparison, the export of
tomato to outside of the focus region has risen steeply in the recent years
(after 2011), while the imports generally indicate a downward trend.
Therefore, invasions from pests such as \tuta{} can have a huge negative
impact on the socioeconomic fabric of this region.  %% A novel ranking-based
%% inference was used to establish that trade was indeed a driving factor in the
%% rapid spread of the pest in this region.
%% In general, modeling the multi-pathway dispersal of pests such as \tuta{}
%% is a challenging task due to inadequate understanding of the complex
%% interconnected food system. To add to the
%% problem, most countries ended up not being prepared for the infestation,
%% either due to lack of awareness of the pest or the sheer speed of invasion.
%% Absence of quality incidence records makes callibration and validation
%% hard. Some of these problems were highlighted in the modeling
%% effort by 

%% In recent years, there has been a thrust towards integrated modeling
%% approaches to understand invasive species dynamics. 
%%
\paragraph{Literature survey.} Multi-pathway models are being increasingly
used to study the role of invasive species dispersal.
Douma~et~al.~\cite{douma2016pathway} survey the literature categorizing
various efforts into flow-based pathway models and agent-based models.
Robinet~et~al.~\cite{robinet2009role} show that the distribution and spread
pattern of the pinewood nematode in China is strongly correlated with
density of human population and infrastructure such as railways and river
ports.  Carrasco~et~al.~\cite{carrasco2010unveiling} combine spatially
explicit models of human-mediated spread with a phenology model to
incorporate population dynamics of the pest (western corn rootworm).
%%
%% Two types of long-distance dispersals
%% are considered -- domestic and international. The domestic mode is modeled
%% as a flow network between cities using a gravity model approach.
%%
Nopsa~et~al.~\cite{nopsa2015ecological} use a network science approach to
studying the role of transport and storage infrastructure in the spread of
pests and pathogens of wheat. Sutrave~et~al.~\cite{sutrave2012identifying}
use a time-varying network model to study the spread of Soybean rust.  Our
model is in part motivated by the hybrid approaches used in the study of
infectious diseases of humans and livestock.
Bradhurst~et~al.~\cite{bradhurst2015hybrid} study the spread of foot and
mouth disease in livestock by using an aggregate population-level model to
capture within-herd spread and an individual-based model for between herd
spread. A similar approach is used by Yang~et~al.~\cite{yang2016} to
forecast influenza outbreaks. 
%%
%% They use a patch network model where a
%% compartmental model is used to simulate intra-locale spread and a gravity
%% model based approach is used for inter-locale spread of flu in the
%% neighborhoods of New~York. 
%%
Although there is a general consensus that vegetable and seedling trade is
a primary driver of \tuta{} spread, previous modeling efforts have
exclusively focused on ecological aspects. Several
studies~\cite{desneux2010biological,tonnang2015identification} provide risk
maps using CLIMEX and take additional factors into account.
Guimapi~et~al.~\cite{guimapi2016modeling} used a cellular automata approach
to capture the global spread of the pest by factoring in temporal
variations and spatial distribution of vegetation, temperature, and tomato
production. A precursor to this work~\cite{venkatramanan2019modeling}
modeled the seasonal production and trade of tomato in Nepal to study the
role of trade in the spread of \tuta{} in Nepal using gravity model and
network dynamics.


%% The modeling framework developed in this paper can
%% be used to study the spread of \tuta{} in other regions. However, it is
%% possible that additional factors need to be accounted for. 
%% It can also be applied to other pests. However, this would require first
%% assessing the importance of each pathway based on evidence: diapause and flying
%% capacity, spread by human mobility or trade, etc.  



Calibration and sensitivity analysis of large-scale, high-resolution models
is a challenging task. Methods based on machine learning surrogates are
being increasingly used to accomplish this efficiently for complex
agent-based models~\cite{lamperti2018agent}. 
%%
\paragraph{Challenges and limitations.} Modeling emerging invasions is
particularly challenging. Limited data on incidence and understanding of
the underlying dynamics makes it nearly impossible to calibrate and
validate the models.   We have had to simplify or ignore some of the
processes that might significantly influence the spread.  For example, our
model uses monthly production as a surrogate for infectiousness of a cell.
Complex phenology models can be used instead (as in
Carrasco~et~al.~\cite{carrasco2010unveiling}). While such a model for
\tuta{} would be useful, this would add to the complexity of the
model~\cite{robinet2012suite}.

Since our focus region spans multiple countries, identifying and collecting
data for each country was a tedious process. For many countries, data had
to be collected (or even inferred) from several publications and reports
(Table~\ref{S:tab:countryData}). Further, these datasets were misaligned in
time and spatial resolution.  It is important to account for heterogeneity
in production, consumption, awareness, cultural factors, etc. both within
and between countries.  Some countries are technologically more advanced
than others, which manifests as differences in yield, crop loss, trade
infrastructure, pest awareness and preparation for
invasion~\cite{early2016global}. More pathways might have to be accounted
for as well (seedling trade, greenhouse cultivation, etc.).

%%
%% multi-pathway model incorporated a multitude of
%% datasets-- production, consumption, trade dynamics, climate, biology, etc.
%% On the other hand, 
%% To capture the
%% multi-pathway dynamics complexity of
%% invasion dynamics demands the use of several datasets spanning mulitple
%% domains.
%% Hence, we have had to simplify or ignore some of
%% the processes that might significantly influence the spread.
%% amid data scarcity and
%% heterogeneity. makes modeling has forced
%% us to simplify or ignore some of the processes that might
%% significantly influence the spread. 
%%

In particular, it is hard to model human assisted spread owing to lack of
seasonal trade data. Production is dependent not only on the host, climate,
and geography, but also on people's preferences and market demand. To
determine outflows and inflows for each locality, we had to identify major
ports for imports and exports as well as estimate fraction of production
which was used for processing. This data was available only for a few
countries.  The farm--market-consumer interactions (local human-mediated
spread) involves various actors such as farmers, wholesalers, retailers,
wet markets, supermarkets, etc. Modeling this is a challenge in itself. If
data on actual flow of vegetables is provided, the gravity model can be
improved or replaced by more sophisticated approaches. Also, the
relationship between long-distance invasion risk and trade volume is hard
to determine. While a direct relationship between volume and risk is
plausible, whether the relation is linear (as assumed by our model) is not
clear.

%% Therefore, data came from disparate sources, and were misaligned in time
%% and spatial resolution. Production and trade data from FAOSTAT also had
%% gaps in them.  
%% The fidelity of models such as the one presented here crucially depends on
%% the availability of quality data as well as a good understanding of the
%% processes involved. 
%% Another example is the modeling of consumption.
%% It is possible that there is lot of variation in consumption within a
%% country~\cite{wijk2007} as well as across seasons. Even at the country
%% level, data is available for only half of the countries. Also, we did not
%% find any correlation between consumption and GDP or tomato production
%% (correlation $< 0.01$).  to model accurately the growth
%% of the pest under specific conditions including the tritrophic interactions
%% concerning the pest, host, and its indigenous predators
%% Rebaudo~et~al.~\cite{rebaudo2011}, for example, use a complex
%% agent-based model to study just the interaction between farmers of two
%% villages in the context of the potato moth in Ecuador. 

%% Even at the country level, data is available for only half of the
%% countries. Also, we did not find any correlation between consumption and
%% GDP or tomato production ($p< 0.01$).  Without adequate data, it is
%% difficult to account for such heterogeneity.

\paragraph{Conclusion.} We developed a generic networked modeling framework
to understand invasive species spread accounting for self- and
human-mediated pathways. Machine learning techniques were used to address
data sparsity and model uncertainties. The model was applied to study the
spread of \tuta{} in the region of Southeast Asia. Our results suggest that
trade of host plants plays an important role in the spread of the pest.
Monitoring and control of this pathway can significantly mitigate the
spread. The methodology uses open-source datasets and can be applied to
other invasive species and host crops. Besides invasive species spread,
other potential applications for this work include studies of natural or
human-initiated disasters, climate change and optimization of food flows.

%% applying this generic modeling approach to other study regions such as
%% North America or Australia would require taking additional
%% factors into account. For example, seedling trade could be an important pathway. From a
%% production perspective, it is becoming important to factor in protected
%% cultivation methods, which have enabled farmers to extend the growing
%% season.
%% However, the developed framework is modular and extensible. Given
%% high-resolution accurate datasets and a better knowledge of the processes,
%% individual modules can be replaced with more sophisticated modeling
%% approaches.  
\paragraph{Data availability.} The authors declare that the data supporting the
findings of this study are available within the paper and its Supplementary
Information file, or from the authors upon reasonable request.

\paragraph{Acknowledgments}
This work was supported in part by the United States Agency for
International Development under the Cooperative Agreement NO.
AID-OAA-L-15-00001 Feed the Future Innovation Lab for Integrated Pest
Management, DTRA CNIMS Contract HDTRA1-11-D-0016-0001, NSF BIG DATA Grant
IIS-1633028, NSF DIBBS Grant ACI-1443054, NIH Grant 1R01GM109718 and NSF
NRT-DESE Grant DGE-154362.  We are grateful to Yousuf Mian, Nguyen Van Hoa,
and Kimhian Seng for their help with obtaining country-specific information
on production, trade, and pest incidence. We thank Richard Beckman and
Irene Eckstrand for useful discussions on model design and paper
organization.

\paragraph{Author contributions.}
AA defined the scope of the
research. AA, JM, TB, MRC collected and interpreted data.
AA, MM conceived and designed the
experiments. JM, AA and YYC performed the
analysis. HM, ND, TB and RM provided assistance in interpreting the
results. AA and JM wrote the paper with significant inputs from
MRC and YYC. AA supervised the research. All authors discussed the
results and commented on the manuscript.

\bibliographystyle{abbrv}
\bibliography{refs}
%%
\end{document}


DOAS
============================================================
VN starting cells not true to radial spread


Random Forest (RF), proposed by Breiman(2001), is a nonparametric ensemble learning method for classification and regression. This method overcomes decision trees’ problem of overfitting by constructing a multitude of decision trees. In order to allow chances to multiple strong predictors to split the trees, each tree uses only a subset of the predictors. In regression, RF predicts the response variable by averaging the predicted values from all trees. Overall, RF provides more advantages over other decision trees such as CART and Bagging, especially for accuracy and robustness.

%% .  the start
%% time step of simulation plays an important role (Figure~\ref{fig:rfShort})
%% as the spread pattern is mostly radial.
%% This is mainly because the spread pattern is radial in turn is
%% dictated by scaling factors $\asd$ and $\afm$, and start month for the
%% simulation. The gravity model parameters $\beta$ and $\kappa$ do not
%% influence this model as there is no long-distance spread. We also note that
%% the monthly production template (uniform or seasonal) does not affect the
%% spread behavior. This is mainly because the values of short-distance
%% scaling factors~$\asd$ or (and)~$\afm$ are relatively high to negate the
%% effect of heterogeneity in monthly production.
%% 
%% For the remaining ranges of~$(\mooreRange,\ell)$, we observe a richer set of
%% emergent behaviors. While for very high values of~$\ald$ and~$\afm$, we
%% again see Class~A spread pattern, Class~B dominates and is strongly
%% influenced by monthly production template, \jmcomment{and?} start month. While~$\beta$ is a
%% significant factor in the spread, the outcome is not highly sensitive
%% to~$\kappa$.


%% Our analysis strongly indicates that both short distance spread and trade
%% are important pathways in the spread dynamics. For the parameter set with
%% the best fit, the similarity was close to~$6$, i.e., on an average,
%% simulation output matched six out of eight locations.   This is in some sense
%% expected due to long distance jumps facilitated by human-assisted spread.
%% It also means that \tuta{} can achieve the same (or greater) range
%% expansion with considerably less flying capacity and population growth.

%% While our results highlight the role of human-mediated spread, it is
%% possible that this pathway is more important for the spread than it
%% appears.  First of all, vegetable production data for Bangladesh~\cite{spam}
%% indicates that almost all cells have non-zero production of hosts of
%% \tuta{}. Therefore, there is a contiguous landscape of suitable areas for
%% the pest to spread naturally. But in general this may not be the case; the
%% only way two locations can be connected is by trade or travel pathways. One
%% obvious example is two land masses separated by sea.  

%%
%%\paragraph{Sensitivity analysis.} 
%%We assessed the role of the model
%%parameters (Table~\ref{tab:parameters}) by restricting our attention.
%%\begin{itemize}
%%    \item machine learning surrogates
%%    \item CART something, default settings
%%\item If we focus on all
%%instances of parameters for which similarity is greater than~$12$ ($75\%$
%%match),
%%\end{itemize}
%%
%%
